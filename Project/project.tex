\documentclass{article}

\usepackage{amsmath,amsfonts,stmaryrd,amssymb} 
\usepackage[backend=bibtex,natbib=true,sorting=none]{biblatex}

\addbibresource{MT.bib}

\title{Ultrafast manipulation of magnetic systems} 

\author{Juan Manuel Losada\\[1cm]{\small Supervisor: Alireza Qaiumzadeh}}

\date{\today} 

%----------------------------------------------------------------------------------------

\begin{document}

\maketitle % Print the title

\section{Introduction} % Numbered section

The discovery of the giant magnetoresistance effect by P. Grünberg \cite{Binasch1989} and A. Fert \cite{Baibich1988} in the 80s opened the gate for a new field of study known as spintronics, in which electron spin is exploited in improved electronic devices. Magnetoresistive random-access memory (MRAM) is an example of a new type of memory in which data is stored in magnetic elements \citep{Akerman2005}. Newer techniques include thermal assisted switching (TA-MRAM) \citep{Bandiera2015} and spin transfer torque (SPRAM) \citep{Kawahara2012}.

However, so far in most current spintronic devices antiferromagnets play only a minor role. During several decades after the discovery of antiferromagnetics, these materials were perceived as useless from a practical point of view. On contrast, ferromagnets have been widely studied historically for its technical applications. However, the development of information technology demands devices with high storage density, high energy efficiency and high write-read speeds. Therefore, controlling magnetically ordered systems on subpicosecond timescales is currently a widely studied area. Antiferromagnets aim to complement or even replace ferromagnets due to their higher magnetic dynamics, their rigidity to external magnetic field, and the absence  of stray field.

\section{Project description}

In this project we want to investigate the magnetic dynamics of antiferromagnets in both insulating and metallic phases as well as in the novel two dimensional ferromagnets. The goal is to propose efficient techniques to manipulate the magnetic state of such materials. A possible way to do this is to study the interaction between light and the exchange interaction and the Dzyaloshinskii-–Moriya interaction \cite{Mentink2015}. For ferromagnetic materials it has already been shown that short laser pulses can induce an effective magnetic field that can reach an amplitude of a few Tesla (\cite{Qaiumzadeh2016}, \cite{Qaiumzadeh2013}). The results of this project could lead not only to applications for ultra-fast writing devices, but also to a better understanding of the ultrafast dynamics of many-body systems which is still an open topic. 

Additionally, we want to study the magnetic properties of topological insulators with special interest in the mathematical formulation of topological invariants. Topological materials are of great interest and promises potential applications for spintronics.

\printbibliography 

\end{document}
