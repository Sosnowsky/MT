\chapter{Floquet Theory}
\label{APE}
In this appendix we will state and proof the Floquet theorem and employ the Floquet formalism to derive a perturbative analysis of high frequency time periodic Hamiltonians. This formalism been widely used recently in many-body driven systems (\cite{Desbuquois2017}, \cite{Bordia2017}, \cite{Gorg2018}).

We will start with a periodic Hamiltonian:

\begin{equation}
\hat{H}(t) = \hat{H}(t+T)
\end{equation}

The Floquet theorem states that the eigenstates for the time evolution operator for one period, $\hat{U}(t+T,t)$ can be written as:

\begin{equation}
\ket{\psi_n(t)} = e^{-i \epsilon_n t}\ket{u_n(t)}
\end{equation}

Where $\ket{u_n(t+T)} = \ket{u_n(t)}$ is a periodic function, called Floquet mode, and $\epsilon_n$ is a real number known as the quasienergy. To show this, let $a_n(t)$ be the eigenvalue of $\ket{\psi_n(t)}$ under $\hat{U(t+T,t)}$, that is $\hat{U(t+T,t)}\ket{\psi_n(t)} = a_n(t)\ket{\psi_n(t)}$, then by multiplying this equation by $\hat{U}(t',t)$ from the left and using the periodicity of the Hamiltonian in $\hat{U}(t',t) = \hat{U}(t'+T,t+T)$ we obtain:

\begin{align*}
\hat{U}(t'+T,t) \ket{\psi_n(t)} &= a_n(t)\ket{\psi_n(t')} \rightarrow \\
\hat{U}(t'+T,t) \hat{U}(t,t')\hat{U}(t',t) \ket{\psi_n(t)} &= a_n(t)\ket{\psi_n(t')} \rightarrow \\
\hat{U}(t'+T,t') \ket{\psi_n(t')} &= a_n(t)\ket{\psi_n(t')}
\end{align*}

Which means that $a_n(t') = a_n(t)$, so the eigenvalue does not depend on time, and since it is an eigenvalue of an unitary operator it can be written as $a_n = a_n(t) = e^{-i\epsilon_nT}$ for certain real number $\epsilon_n$. Therefore, we have 

\begin{align*}
\ket{\psi_n(t+T)} &= e^{-i\epsilon_nT} \ket{\psi_n(t)} \rightarrow \\
\ket{\psi_n(t)} &= e^{-i\epsilon_nt} \ket{\u_n(t)}
\end{align*}

For $\ket{\u_n(t)} = e^{i\epsilon_nt} \ket{\psi_n(t)} = \ket{\u_n(t+T)}$, and this proves the theorem.