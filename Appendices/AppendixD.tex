\chapter{Tight binding model in the honeycomb lattice}
\label{APD}

Graphene, as a representative two dimensional solid with honeycomb lattice structure has some interesting electrical properties induced by its lattice structure. Graphene is a zero gap semiconductor, meaning that its conduction and valence bands touch each other at the $\bs{K}$ points in the momentum space. Around these points the electrons have a linear dispersion relation, resembling that of massless Dirac fermions. In the following we will show this in detail.

Using the same notation introduced in \ref{AP1A} we can introduce the fermionic operators $\hat{a}^\dagger_i$ and $\hat{b}^\dagger_j$ which create and electron on the A and B site receptively, at position $\bs{R}_i$ or $\bs{R}_i$. Then, the tight binding Hamiltonian considering only NN hopping is:

\begin{equation}
\hat{H} = -t\sum_{i \in A}\sum_{\bs{\delta}} (\hat{a}^\dagger_i\hat{b}_{i+\bs{\delta}} + \hat{b}^\dagger_{i+\bs{\delta}}\hat{a}_i)
\end{equation}

Where $i$ labels sites in sublattice A and $\bs{\delta}$ are the NN vectors, we make an abuse of notation by summing $i+\bs{\delta}$ referring, of course, to the site at position $\bs{R}_i+\bs{delta}$. 
