\begin{section}{Including next nearest neighbor hopping}

In the previous sections we have seen that the second order perturbation effective Hamiltonian will contain the following terms:

\begin{itemize}
	\item Exchange interaction $J_{ij} \bs{S}_i \bs{S}_j$, arising from the kinetic hopping - kinetic hopping terms.
	\item DMI $\bs{D}_{ij} \bs{S}_i \times \bs{S}_j$, arising from the kinetic hopping - SOI hopping terms.
	\item Anisotropic or pseudodipolar interaction $\bs{S}_i \bs{\Gamma}_{ij} \bs{S}_j$, arising from the SOI hopping - SOI hopping terms.
\end{itemize}

And all these coupling factors will be renormalized by the laser field. This renormalization depends only on the field and the sites $i$, $j$, it does not depend on the nature of the interaction. With this in mind we will introduce a model with additional NNN hopping terms: 

\begin{equation}
\label{BigHam}
\hat{H} = -\sum_{\langle i,j \rangle, \sigma, \sigma'}(\delta_{\sigma, \sigma'} t_1 + \bs{\Delta}_{1,ij} \bs{\sigma}_{\sigma, \sigma'})\hat{c}_{i \sigma}^\dagger \hat{c}_{j \sigma'} - 
	\sum_{\langle \langle i,j \rangle \rangle, \sigma, \sigma'}(\delta_{\sigma, \sigma'} t_2 + \bs{\Delta}_{2,ij} \bs{\sigma}_{\sigma, \sigma'})\hat{c}_{i \sigma}^\dagger \hat{c}_{j \sigma'} + 
	\text{U}\hat{D}
\end{equation}

Where sum over next-nearest neighbors is denoted by $\langle \langle i j \rangle \rangle$. This Hamiltonian is the same we studied in section \ref{Section3HubbardSOI} plus a NNN hopping and SOI term. The hopping amplitudes thus are:

\begin{equation}
t_{ij}^{\sigma \sigma'} = \begin{cases}
	(\delta_{\sigma \sigma'}t_1+\bs{\Delta}_{1,ij} \bs{\sigma}_{\sigma, \sigma'}) & \text{for } i, j \text{ nearest neighbors} \\
	(\delta_{\sigma \sigma'}t_2+\bs{\Delta}_{2,ij} \bs{\sigma}_{\sigma, \sigma'}) & \text{for } i, j \text{ next nearest neighbors} \\
	0 & \text{ otherwise}
\end{cases} \quad
\end{equation}

The effective Hamiltonian will be the same as in section \ref{Section3HubbardSOI}, with the corresponding NNN terms. This is because in second order perturbation NN hopping terms do not mix with NNN hopping terms. Thus, the effective spin Hamiltonian will be:

\begin{align}
\hat{H}_{\text{eff}}(t) = &\sum_{\langle i,j \rangle} \left\{ J_{1,ij}(t)\bs{S}_i\bs{S}_j + \bs{D}_{1,ij}(t) \bs{S}_i \times \bs{S}_j + \bs{S}_i\bs{\Gamma}_{1,ij}(t)\bs{S}_j\right\} + \nonumber \\
&\sum_{\langle \langle i,j \rangle \rangle} \left\{ J_{2,ij}(t)\bs{S}_i\bs{S}_j + \bs{D}_{2,ij}(t) \bs{S}_i \times \bs{S}_j + \bs{S}_i\bs{\Gamma}_{2,ij}(t)\bs{S}_j\right\}
\end{align}

Where:

\begin{align}
J_{n,ij}(t) &= t_n^2\mathcal{M}(\alpha_{ij}, \text{U}, \omega, t) \label{JGeneral} \\
\bs{D}_{n,ij}(t) &= 2it_n \bs{\Delta}_{n,ij}\mathcal{M}(\alpha_{ij}, \text{U}, \omega, t) \label{DGeneral} \\
\Gamma_{n,ij}^{\alpha \beta}(t) &= (\delta_{\alpha \beta} \bs{\Delta}_{n,ij}^2 - 2\Delta_{n,ij}^\alpha\Delta_{n,ij}^\beta )\mathcal{M}(\alpha_{ij}, \text{U}, \omega, t) \label{GammaGeneral}
\end{align}

NNN interaction coupling factors will be normalized according to $\mathcal{M}(\alpha_{ij}, \text{U}, \omega, t)$, where $i$, $j$ are NNN sites, whereas NN interaction coupling factors will be normalized according to $\mathcal{M}(\alpha_{ij}, \text{U}, \omega, t)$ where $i$, $j$ are NN sites. Therefore the dependence on the field will differ. For example, if we take the time independent approximation \ref{MFactorApprox}, and use $\lambda = \pm 1$ for circular polarized light, and $\alpha_{ij} = \pm\frac{\mathcal{E}|\bs{R}_{ij}|}{a\sqrt{2}}$, where $\mathcal{E} = \frac{eaE_0}{\omega}$. Then the exchange interaction will be:

\begin{align*}
J_{1,ij} &= J_1^0 + t_1^2 \frac{\mathcal{E}^2}{2} \left( \frac{1}{\text{U}+\omega} + \frac{1}{\text{U}-\omega} \right) = J_1^0 + J_1^0 \frac{\mathcal{E}^2}{2} \left( \frac{1}{1+\frac{\omega}{\text{U}}} + \frac{1}{1-\frac{\omega}{\text{U}}} \right) \\
J_{2,ij} &= J_2^0 + t_2^2 \frac{3\mathcal{E}^2}{2} \left( \frac{1}{\text{U}+\omega} + \frac{1}{\text{U}-\omega} \right) = J_2^0 + J_2^0 \frac{3\mathcal{E}^2}{2} \left( \frac{1}{1+\frac{\omega}{\text{U}}} + \frac{1}{1-\frac{\omega}{\text{U}}} \right) \\
\end{align*}

Where $J_n^0 = \frac{t_n^2}{\text{U}}$ and where we used that $|\bs{R}_{ij}| = a$ for NN and $|\bs{R}_{ij}| = \sqrt{3}a$ for NNN in a honeycomb lattice. The other coupling factors in \ref{DGeneral} and \ref{GammaGeneral} can be approximated in the same way.

In subsection (add reference!!!!!!!!!!!) we will investigate this numerically. The reason for this is that in a hopping process, the electron picks up a phase $e^{ie\bs{R}_{ij}\bs{A}(t)}$ (assuming flat field approximation). This phase will be larger for NNN hopping than for NN hopping and this will therefore translate in the corresponding spin interactions.

Next we will show several well-known models which are described by \ref{BigHam}.

\begin{subsection}{Kane-Mele-Hubbard model}

The first model to describe topological insulators was introduced by Kane and Mele \cite{Kane2005} to describe quantum spin Hall effect in graphene. In a honeycomb lattice time reversal symmetry and inversion symmetry allow only next-nearest neighbor spin orbit coupling, which is known as intrinsic spin orbit coupling. In these circumstances the system can be modeled by the Kane-Mele-Hubbard model:

\begin{align}
\hat{H}_{\text{KMH}} &= -t_1\sum_{\langle i j \rangle \sigma} \hat{c}^{\dagger}_{i\sigma}\hat{c}_{j\sigma} + i\Delta \sum_{\langle \langle i j \rangle \rangle \sigma \sigma'} \hat{c}^{\dagger}_{i\sigma} \nu_{ij} \sigma^z_{\sigma \sigma'} \hat{c}_{j\sigma'} + \text{U}\hat{D}
\end{align}

Where $\Delta$ is the intrinsic spin orbit coupling constant. $\nu_{ij}=\pm 1$ depending on whether the electron traversing from $i$ to $j$ makes a right ($+1$) or a left turn ($-1$). This Hamiltonian is described by \ref{BigHam} if we set $\bs{\Delta}_{1,ij} = 0$, $t_2 = 0$ and $\bs{\Delta}_{2,ij} = -i\Delta \nu_{ij} \hat{e}_z$. Then, the effective spin model will be:

\begin{equation}
\hat{H}_{\text{KMH}}^{\text{eff}}(t) = \sum_{\langle i,j \rangle} J_{ij}(t) \bs{S}_i \bs{S}_j + \sum_{\langle \langle i,j \rangle \rangle} \bs{S}_i \bs{\Gamma}_{ij}(t) \bs{S}_j 
\end{equation}

With and:

\begin{align*}
J_{ij}(t) &= t_1^2\mathcal{M}(\alpha_{ij}, \text{U}, \omega, t) \\
\bs{\Gamma}_{ij}(t) &= \Delta^2 \text{diag}(-1,-1,1)\mathcal{M}(\alpha_{ij}, \text{U}, \omega, t)
\end{align*}

according to \ref{JGeneral} and \ref{GammaGeneral}. Notice that:

\begin{equation*}
\bs{S}_i \bs{\Gamma}_{ij}(t) \bs{S}_j = \Delta^2 \mathcal{M}(\alpha_{ij}, \text{U}, \omega, t) \left(S_i^zS_j^z - S_i^xS_j^x - S_i^yS_j^y\right)
\end{equation*}

This describes a type of anisotropic exchange interaction known as XXZ Heisenberg model for next nearest neighbors. The same spin model is obtain in \cite{Rachel2010} without the laser perturbation. If the laser field is not too strong so that we can assume $\mathcal{M}(\alpha_{ij}, \text{U}, \omega, t) > 0$, then $\Gamma^{zz}_{ij}(t) > 0$. Therefore we see that this interaction favors antiferromagnetic order in the $\hat{e}_z$ direction and ferromagnetic order in the $\hat{e}_x-\hat{e}_y$ plane. The exchange interaction $J_{ij}(t)$ will favor antiferromagnetic order for nearest neighbors, so that next nearest neighbors will tend to be aligned. Therefore, $\Gamma^{zz}_{ij}(t)$ will compete against $J_{ij}(t)$ in the $\hat{e}_z$ direction. In the $\hat{e}_x-\hat{e}_y$ plane, $\bs{\Gamma}_{ij}(t)$ will favor ferromagnetic order between next nearest neighbors, which adds to the effect of $J_{ij}(t)$. In general the strength of the exchange interaction will be larger and the net effect of $\bs{\Gamma}_{ij}(t)$ will be a tilting of the spins towards the $\hat{e}_x$-$\hat{e}_y$ plane.

\end{subsection}

\end{section}