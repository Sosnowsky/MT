\chapter{Spin wave theory of antiferromagnets}
\label{SpinWave}

In this Chapter we will study the magnon dispersion for the effective spin Hamiltonian obtained in the last two Chapters. The standard way of doing this is to use the Holstein-Primakoff transformation, which maps S-spin operators on a lattice to bosonic creation/annihilation operators. The linear approximation of this transformation leads to a bosonic Hamiltonian which can be diagonalized to obtain the magnon dispersion relation. Keeping higher order terms in this transformation is also useful to study scattering effects between magnons.

We will start studying the simplest model of spin interaction, the Heisenberg model. In the following sections we will add terms such as the DMI and the anisotropic NNN term derived in \ref{MKMHeff}.

\section{Heisenberg antiferromagnet}

We will start with the Heisenberg antiferromagnet:

\begin{equation}
\hat{H} = J \sum_{\langle i j \rangle} \bs{S}_i \bs{S}_j
\end{equation}

In order to obtain the magnon Hamiltonian for this model we introduce the antiferromagnet Holstein-Primakoff transformation:

\begin{align}
S^+_{Ai} &= \sqrt{2S-\hat{a}^\dagger_i \hat{a}_i} \hat{a}_i \\
S^-_{Ai} &= \hat{a}^\dagger_i\sqrt{2S-\hat{a}^\dagger_i \hat{a}_i}  \\
S^z_{Ai} &= S - \hat{a}^\dagger_i \hat{a}_i \\
S^+_{Bj} &=\hat{b}^\dagger_j\sqrt{2S-\hat{b}^\dagger_j \hat{b}_j} \\
S^-_{Bj} &= \sqrt{2S-\hat{b}^\dagger_j \hat{b}_j} \hat{b}_j \\
S^z_{Bj} &= -S + \hat{b}^\dagger_j \hat{b}_j
\end{align}

Where $A$ and $B$ denote two Ne\'{e}l sublattices and $\hat{a}_i$, $\hat{b}_i$ are bosonic annihilation operators. In the large-S limit we approximate $ \sqrt{2S-\hat{a}^\dagger_i \hat{a}_i} \approx \sqrt{2S}$. Then, the exchange term becomes:

\begin{align*}
\bs{S}_i\bs{S}_j = S_i^zS_j^z + \frac{1}{2}\left(S_i^+S_j^- + S_i^-S_j^+ \right) = S \left( \hat{a}^\dagger_i \hat{a}_i + \hat{b}^\dagger_j \hat{b}_j + \hat{a}_i\hat{b}_j + \hat{a}^\dagger_i\hat{b}^\dagger_j \right) - \hat{a}^\dagger_i \hat{a}_i\hat{b}^\dagger_j \hat{b}_j - S^2
\end{align*}

Now the $S^2$ term is a constant that lowers the energy of the antiferromagnet, therefore is shall not be taken into account to compute the magnon dispersion. The term $\hat{a}^\dagger_i \hat{a}_i\hat{b}^\dagger_j \hat{b}_j$ is zeroth order in $S$ and therefore neglected. We can rewrite the sum over NN as:

\begin{equation}
\sum_{\langle i j \rangle} = 2\sum_{i \in A \delta}
\end{equation}

Where $\delta$ are the NN vectors. Using this, and the Fourier relations:

\begin{align}
\hat{a}_i^\dagger &= \frac{1}{\sqrt{N}} \sum_{\bs{k}} e^{i \bs{k} \cdot \bs{r}_i} \hat{a}_{\bs{k}}^\dagger \\
\hat{b}_j^\dagger &= \frac{1}{\sqrt{N}} \sum_{\bs{k}} e^{i \bs{k} \cdot \bs{r}_j} \hat{b}_{\bs{k}}^\dagger
\end{align}

Where the sum is over $\bs{k}$ in the first Brillouin zone. Using these relations, together with $\sum{i} e^{i(\bs{k}-\bs{k}')\cdot\bs{r}_i} = \delta_{\bs{k}, \bs{k}'}$ we find:

\begin{equation}
\hat{H} = 2JS \sum_{\bs{k} \bs{\delta}} \left( \hat{a}^\dagger_{\bs{k}}\hat{a}_{\bs{k}} + \hat{b}^\dagger_{\bs{k}}\hat{b}_{\bs{k}} + e^{i\bs{k}\bs{\delta}}\hat{a}_{\bs{k}}\hat{b}_{\bs{-k}} + e^{-i \bs{k}\cdot\bs{\delta}}  \hat{a}^\dagger_{\bs{k}}\hat{b}^\dagger_{-\bs{k}} \right)
\end{equation}

At this point it is common to define $\gamma_{\bs{k}} = \sum_\delta e^{i \bs{k} \cdot \bs{\delta}}$. Using $\bs{\delta}_1=\frac{a_0}{2}(1,\sqrt{3})$, $\bs{\delta}_2=\frac{a_0}{2}(1,-\sqrt{3})$, $\bs{\delta}_3=a_0(-1,0)$ and taking $a_0=1$, we can write this as:
\begin{equation}
\gamma_{\bs{k}} = 2e^{\frac{i}{2}k_x}\cos(\frac{\sqrt{3}}{2}k_y)+e^{-ik_x}
\end{equation}
Then denoting $z = \sum_\delta$ we get:

\begin{equation}
\label{MagnonicH}
\hat{H} = 2JS \sum_{\bs{k}} \left\{ z \left( \hat{a}^\dagger_{\bs{k}}\hat{a}_{\bs{k}} + \hat{b}^\dagger_{-\bs{k}}\hat{b}_{-\bs{k}} \right) + \gamma_{\bs{k}}\hat{a}_{\bs{k}}\hat{b}_{\bs{-k}} + \gamma_{-\bs{k}}\hat{a}^\dagger_{\bs{k}}\hat{b}^\dagger_{-\bs{k}} \right\}
\end{equation}

Which can be written in matrix form as:

\begin{equation}
\hat{H}_{\bs{k}} = 2JS\begin{pmatrix} 
z & \gamma_{\bs{k}} \\
\gamma_{-\bs{k}} & z
\end{pmatrix}
\end{equation}

Acting on the spinor $\Psi_{\bs{k}} = \left(\hat{a}_{\bs{k}},\hat{b}_{-\bs{k}}^\dagger \right)^T$. That is, $\hat{H} = \sum_{\bs{k}} \Psi_{\bs{k}} \hat{H}_{\bs{k}} \Psi^\dagger_{\bs{k}}$. In order to diagonalize this Hamiltonian we introduce the Bogoliubov transformation:

\begin{equation}
\label{Bogoliubov}
\begin{pmatrix}
\hat{\alpha}_{\bs{k}} \\
\hat{\beta}_{\bs{k}}
\end{pmatrix} = 
\begin{pmatrix}
u_{\bs{k}} & v_{\bs{k}}^* \\
v_{\bs{k}} & u_{\bs{k}}^*
\end{pmatrix}
\begin{pmatrix}
\hat{a}_{\bs{k}} \\
\hat{b}_{-\bs{k}}^\dagger
\end{pmatrix}
\end{equation}

Where $u_{\bs{k}}$ and $v_{\bs{k}}$ are complex functions satisfying $\abs{u_{\bs{k}}}^2-\abs{v_{\bs{k}}}^2 = 1$ in order to perserve the bosonic commutation relations. The inverse transformation is given by:

\begin{equation}
\label{BogoliubovInv}
\begin{pmatrix}
\hat{a}_{\bs{k}} \\
\hat{b}_{-\bs{k}}^\dagger
\end{pmatrix} = 
\begin{pmatrix}
u_{\bs{k}}^* & -v_{\bs{k}}^* \\
-v_{\bs{k}} & u_{\bs{k}}
\end{pmatrix}
\begin{pmatrix}
\hat{\alpha}_{\bs{k}} \\
\hat{\beta}_{\bs{k}}^\dagger
\end{pmatrix}
\end{equation}

Inserting this in \ref{MagnonicH} and ignoring constant terms we get:

\begin{align}
\hat{H} &= 2JS \sum_{\bs{k}} \left\{ \left( \hat{\alpha}_{\bs{k}}^\dagger \hat{\alpha}_{\bs{k}} +\hat{\beta}_{\bs{k}}^\dagger \hat{\beta}_{\bs{k}} \right) \left( z(\abs{u_{\bs{k}}}^2+\abs{v_{\bs{k}}}^2) -\gamma_{\bs{k}}u_{\bs{k}}^*v_{\bs{k}}^* - \gamma_{-\bs{k}}u_{\bs{k}}v_{\bs{k}} \right) + \right. \nonumber \\
&+ \left.  \hat{\alpha}^\dagger_{\bs{k}} \hat{\beta}_{\bs{k}} \left( -2zu_{\bs{k}}v_{\bs{k}}^* + \gamma_{\bs{k}}(v_{\bs{k}}^*)^2+\gamma_{-\bs{k}}u_{\bs{k}}^2 \right) + \hat{\beta}_{\bs{k}}^\dagger\hat{\alpha}_{\bs{k}} \left( -2zu_{\bs{k}}^*v_{\bs{k}} + \gamma_{\bs{k}}(u_{\bs{k}}^*)^2+\gamma_{-\bs{k}}v_{\bs{k}}^2 \right) \right\} \label{BeforeBogo}
\end{align}

Therefore, in order to erase the non diagonal terms we impose:
\begin{equation}
\label{condition}
-2zu_{\bs{k}}v_{\bs{k}}^* + \gamma_{\bs{k}}(v_{\bs{k}}^*)^2+\gamma_{-\bs{k}}u_{\bs{k}}^2 = 0
\end{equation}
Notice that for the $\hat{\beta}_{\bs{k}}^\dagger\hat{\alpha}_{\bs{k}}$ term $-2zu_{\bs{k}}^*v_{\bs{k}} + \gamma_{\bs{k}}(u_{\bs{k}}^*)^2+\gamma_{-\bs{k}}v_{\bs{k}}^2=0$ is the same equation. This, together with $\abs{u_{\bs{k}}}^2-\abs{v_{\bs{k}}}^2 = 1$ determines $u_{\bs{k}}$ and $v_{\bs{k}}$. To see this, first notice that an overall phase in both $u_{\bs{k}}$ and $v_{\bs{k}}$ leads to the same physical state. We can use this fact to impose that $v_{\bs{k}}$ is real, and we can use $\abs{u_{\bs{k}}}^2-\abs{v_{\bs{k}}}^2 = 1$ to parametrize $u_{\bs{k}}$ and $v_{\bs{k}}$ in the following way:

\begin{align*}
u_{\bs{k}} &= \cosh(\frac{\theta_{\bs{k}}}{2})e^{i \xi_{\bs{k}}} \\
v_{\bs{k}} &= \sinh(\frac{\theta_{\bs{k}}}{2})
\end{align*}

Where $\theta_{\bs{k}}$ and $\xi_{\bs{k}}$ are real functions to be determined. Introducing these relations in \ref{condition}:

\begin{equation}
- z\sinh(\theta_{\bs{k}}) +\gamma_{\bs{k}}e^{-i \xi_{\bs{k}}}\frac{ \cosh(\theta_{\bs{k}})-1}{2} +\gamma_{-\bs{k}}e^{i\xi_{\bs{k}}}\frac{ \cosh(\theta_{\bs{k}})+1}{2} = 0
\end{equation}

By taking $\xi_{\bs{k}}$ to be the phase of $\gamma_{\bs{k}}$, i.e. $\gamma_{\bs{k}} = \abs{\gamma_{\bs{k}}} e^{i\xi_{\bs{k}}}$ we get a real equation on $\theta_{\bs{k}}$. Altogether we find:

\begin{align}
\gamma_{\bs{k}} &= \abs{\gamma_{\bs{k}}} e^{i\xi_{\bs{k}}} \\
\tanh(\theta_{\bs{k}}) &= \frac{\abs{\gamma_{\bs{k}}}}{z}
\end{align}

This erases the non diagonal terms in \ref{BeforeBogo} leading to a diagonal term:
\begin{equation}
z\cosh(\theta_{\bs{k}})-\abs{\gamma_{\bs{k}}}\sinh(\theta_{\bs{k}}) = \sqrt{z^2-\abs{\gamma_{\bs{k}}}^2}
\end{equation}

Then, the dispersion relation is:

\begin{align}
\hat{H} &= \sum_{\bs{k}}\left( \hat{\alpha}_{\bs{k}}^\dagger \hat{\alpha}_{\bs{k}} +\hat{\beta}_{\bs{k}}^\dagger \hat{\beta}_{\bs{k}} \right) \epsilon_{\bs{k}} \\
\epsilon_{\bs{k}} &= 2JS \sqrt{z^2-\abs{\gamma_{\bs{k}}}^2}
\end{align}

\section{Adding DMI}

Let's now consider

\begin{equation}
\hat{H} = J \sum_{\langle i j \rangle} \bs{S}_i \bs{S}_j + D \sum_{\langle \langle i j \rangle \rangle} \nu_{ij}\hat{e}_z \bs{S}_i \times \bs{S}_j
\end{equation}

The NNN DMI term will couple spins within the same sublattice. Using the HP transformation:

\begin{equation}
\hat{e}_z \bs{S}_i \times \bs{S}_j = \frac{i}{2}(S_i^+S_j^--S_i^-S_j^+) = \begin{cases}
             iS(\hat{a}_j^\dagger\hat{a}_i-\hat{a}_i^\dagger\hat{a}_j),  & \text{for } (i,j) \in A \\
             iS(\hat{b}_i^\dagger\hat{b}_j-\hat{b}_j^\dagger\hat{b}_i),  & \text{for } (i,j) \in B
       \end{cases} \quad
\end{equation}

Now we will make the sum over NNN in the following way:

\begin{equation}
\sum_{\langle \langle i j \rangle \rangle} = \sum_{i\in A \bs{\delta}^{NNN}} +  \sum_{i\in B \bs{\delta}^{NNN}}
\end{equation}

Where $\bs{\delta}^{NNN}$ are the NNN vectors:
\begin{align*}
\bs{\delta}^{NNN}_1 &= \delta_1-\delta_3 = \frac{1}{2}(3, \sqrt{3}) \\
\bs{\delta}^{NNN}_2 &= \delta_2-\delta_1 = (0, -\sqrt{3}) \\
\bs{\delta}^{NNN}_3 &= \delta_3-\delta_2 = \frac{1}{2}(-3, \sqrt{3}) \\
\bs{\delta}^{NNN}_4 &= -\bs{\delta}^{NNN}_1 = -\frac{1}{2}(3, \sqrt{3}) \\
\bs{\delta}^{NNN}_5 &= -\bs{\delta}^{NNN}_2 = (0, \sqrt{3}) \\
\bs{\delta}^{NNN}_6 &= -\bs{\delta}^{NNN}_3 = \frac{1}{2}(3, -\sqrt{3}) \\
\end{align*}
Notice that with these definitions, if $i$ is a site on the $A$ sublattice, $\nu_{i,i+\bs{\delta}^{NNN}_a}=+1$ for $a=1,2,3$ and $\nu_{i,i+\bs{\delta}^{NNN}_a}=-1$ for $a=4,5,6$. The opposite relations hold for $i$ in the $B$ sublattice.

 Following the same procedure as before we get and additional magnonic term:

\begin{equation}
D \sum_{\langle \langle i j \rangle \rangle} \nu_{ij}\hat{e}_z \bs{S}_i \times \bs{S}_j = 2JS\sum_{\bs{k}} \left\{ \Delta_{\bs{k}}\hat{a}^\dagger_{\bs{k}}\hat{a}_{\bs{k}} - \Delta_{-\bs{k}}\hat{b}^\dagger_{-\bs{k}}\hat{b}_{-\bs{k}} \right\}
\end{equation}

Where $\Delta_{\bs{k}} = \frac{Di}{2J}\sum_{\bs{\delta}^{NNN}} \nu_{i,i+\delta} (e^{i\bs{k}\cdot\bs{\delta}} -e^{-i\bs{k}\cdot\bs{\delta}})$, for $i$ in the $A$ sublattice. This can be rewritten as:
\begin{equation}
\Delta_{\bs{k}} = -\frac{2D}{J}\left\{ \sin(\bs{\delta}^{NNN}_1 \cdot \bs{k}) + \sin(\bs{\delta}^{NNN}_2 \cdot \bs{k}) + \sin(\bs{\delta}^{NNN}_3 \cdot \bs{k})\right\}
\end{equation}
The full magnonic Hamiltonian now is:

\begin{equation}
\label{MagnonicH2}
\hat{H} = 2JS \sum_{\bs{k}} \left\{ \hat{a}^\dagger_{\bs{k}}\hat{a}_{\bs{k}}(z+\Delta_{\bs{k}}) + \hat{b}^\dagger_{-\bs{k}}\hat{b}_{-\bs{k}}(z-\Delta_{-\bs{k}}) + \gamma_{\bs{k}}\hat{a}_{\bs{k}}\hat{b}_{\bs{-k}} + \gamma_{-\bs{k}}\hat{a}^\dagger_{\bs{k}}\hat{b}^\dagger_{-\bs{k}} \right\}
\end{equation}

Or:

\begin{equation}
\hat{H}_{\bs{k}} = 2JS\begin{pmatrix} 
z + \Delta_{\bs{k}}& \gamma_{\bs{k}} \\
\gamma_{-\bs{k}} & z - \Delta_{-\bs{k}}
\end{pmatrix}
\end{equation}

In this case, applying the transformation \ref{Bogoliubov} leads to the same condition for the coefficients  $u_{\bs{k}}$ and $v_{\bs{k}}$, whereas the energy gets lifted:

\begin{align}
\hat{H} &= \sum_{\bs{k}} \left( \hat{\alpha}_{\bs{k}}^\dagger \hat{\alpha}_{\bs{k}} \epsilon_{\bs{k}}^+ +\hat{\beta}_{\bs{k}}^\dagger \hat{\beta}_{\bs{k}}  \epsilon_{\bs{k}}^- \right) \label{DispWithDMI} \\
\epsilon_{\bs{k}}^\pm &= 2JS \left( \pm \Delta_{\bs{k}} + \sqrt{z^2-\abs{\gamma_{\bs{k}}}^2} \right)
\end{align}

\section{NNN exchange}

Let's now introduce a term:

\begin{equation}
\sum_{\langle \langle i,j \rangle \rangle} \left\{ \Gamma_{xy}(S_i^xS_j^x + S_i^yS_j^y) + \Gamma_zS_i^zS_j^z\right\}
\end{equation}
Notice that taking $\Gamma_{xy} = -\abs{J_2}-\abs{\Gamma_2}$ and $\Gamma_z = -\abs{J_2}+\abs{\Gamma_2}$ describes the NNN exchange interaction and the NNN anisotropic interaction in \ref{MKMHeff}. In terms of bosonic operators we have:
\begin{equation}
\Gamma_{xy}(S_i^xS_j^x + S_i^yS_j^y) + \Gamma_zS_i^zS_j^z = \begin{cases}
             S\Gamma_{xy}(\hat{a}_i^\dagger\hat{a}_j+\hat{a}_j^\dagger\hat{a}_i) - S\Gamma_z(\hat{a}_i^\dagger\hat{a}_i+\hat{a}_j^\dagger\hat{a}_j),  & \text{for } (i,j) \in A \\
             S\Gamma_{xy}(\hat{b}_i^\dagger\hat{b}_j+\hat{b}_j^\dagger\hat{b}_i) - S\Gamma_z(\hat{b}_i^\dagger\hat{b}_i+\hat{b}_j^\dagger\hat{b}_j),  & \text{for } (i,j) \in B
       \end{cases} \quad
\end{equation}
Fourier transform leads to:

\begin{align*}
&\sum_{\langle \langle i,j \rangle \rangle} \left\{ \Gamma_{xy}(S_i^xS_j^x + S_i^yS_j^y) + \Gamma_zS_i^zS_j^z\right\}
= S\sum_{\bs{k}} (\Gamma_{xy}\Delta'_{\bs{k}}-4z\Gamma_z) (\hat{a}^\dagger_{\bs{k}}\hat{a}_{\bs{k}}+\hat{b}^\dagger_{-\bs{k}}\hat{b}_{-\bs{k}}) = \\
&=2JS\sum_{\bs{k}} \tilde{\Gamma}_{\bs{k}}(\hat{a}^\dagger_{\bs{k}}\hat{a}_{\bs{k}}+\hat{b}^\dagger_{-\bs{k}}\hat{b}_{-\bs{k}})
\end{align*}

Where $\Delta'_{\bs{k}} = \sum_{\delta^{NNN}} \left( e^{i\bs{k}\cdot \bs{\delta}^{NNN}} +  e^{-i\bs{k}\cdot \bs{\delta}^{NNN}}\right)$ and $\tilde{\Gamma}_{\bs{k}} = \frac{1}{2J}\left( \Gamma_{xy}\Delta'_{\bs{k}}-4z\Gamma_z \right)$. Therefore, now:

\begin{equation}
\hat{H}_{\bs{k}} = 2JS\begin{pmatrix} 
z + \Delta_{\bs{k}} + \tilde{\Gamma}_{\bs{k}} & \gamma_{\bs{k}} \\
\gamma_{-\bs{k}} & z - \Delta_{-\bs{k}} + \tilde{\Gamma}_{\bs{k}}
\end{pmatrix}
\end{equation}

We obtain the dispersion relation by changing $z \rightarrow z + \tilde{\Gamma}_{\bs{k}}$ in \ref{DispWithDMI}:

\begin{align}
\hat{H} &= \sum_{\bs{k}} \left( \hat{\alpha}_{\bs{k}}^\dagger \hat{\alpha}_{\bs{k}} \epsilon_{\bs{k}}^+ +\hat{\beta}_{\bs{k}}^\dagger \hat{\beta}_{\bs{k}}  \epsilon_{\bs{k}}^- \right) \label{DispWithDMI} \\
\epsilon_{\bs{k}}^\pm &= 2JS \left( \pm \Delta_{\bs{k}} +  \sqrt{(z + \tilde{\Gamma}_{\bs{k}})^2-\abs{\gamma_{\bs{k}}}^2} \right)
\end{align}

