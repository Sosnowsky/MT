\begin{section}{Kane Mele Hubbard model}

The first model to describe topological insulators was introduced by Kane and Mele \cite{Kane2005} to describe quantum spin Hall effect in graphene. In a honeycomb lattice time reversal symmetry and inversion symmetry allow only next-nearest neighbor spin orbit coupling, which is known as intrinsic spin orbit coupling. However, if inversion symmetry is broken by i.e. an external electric field in the $\hat{z}$ direction, a Rashba spin orbit interaction term is allowed between nearest neighbors. In these circumstances the system can be modeled by the Kane-Mele-Hubbard model \cite{Laubach2014}:

\begin{align}
\hat{H}_{KMH} &= -t_0\sum_{\langle i j \rangle \sigma} \hat{c}^{\dagger}_{i\sigma}\hat{c}_{j\sigma} + i\Delta \sum_{\langle \langle i j \rangle \rangle \sigma \sigma'} \hat{c}^{\dagger}_{i\sigma} \nu_{ij} \sigma^z_{\sigma \sigma'} \hat{c}_{j\sigma'} \nonumber \\
&+ i\Delta_R \sum_{\langle i j \rangle \sigma \sigma'} \hat{c}^{\dagger}_{i\sigma} \hat{z}(\bs{\sigma}_{\sigma \sigma'} \times \bs{R}_{ij}) \hat{c}_{j\sigma'} + U\hat{D}
\end{align}

Where sum over next-nearest neighbors is denotes by $\langle \langle i j \rangle \rangle$ and $\Delta$ is the intrinsic spin orbit coupling and $\Delta_R$ is the Rashba spin orbit coupling, which is the same as in the previous section if we choose $\bs{\Delta}_{ij} = i\Delta_R (R_{ij}^y, - R_{ij}^x, 0)$. $\nu_{ij}=\pm 1$ depending on whether the electron traversing from i to j makes a right ($+1$) or a left turn ($-1$). In this case the hopping amplitudes can be written as:

\begin{equation}
\label{HoppHubbSOI}
t_{ij}^{\sigma \sigma'} = \begin{cases}
	\delta_{\sigma \sigma'}t_0- i\Delta_R \hat{z}(\bs{\sigma}_{\sigma \sigma'} \times \bs{R}_{ij}) & \text{for } i, j \text{ nearest neighbors} \\
	-i\Delta \nu_{ij} \sigma_{\sigma \sigma'}^z & \text{for } i, j \text{ next-nearest neighbors} \\
	0 & \text{ otherwise}
\end{cases} \quad
\end{equation}

We can apply \ref{HeffSimplified} with this hopping amplitudes to obtain:

\begin{align*}
\hat{H}_{\text{eff}}(t) &= - \frac{1}{2} \sum_{\langle i,j \rangle, \sigma_1, \sigma_2, \sigma_3, \sigma_4}\hat{c}_{i \sigma_1}^\dagger \hat{c}_{j \sigma_2} \hat{c}_{j \sigma_3}^\dagger \hat{c}_{i \sigma_4} t_{ij}^{\sigma_1 \sigma_2} t_{ji}^{\sigma_3 \sigma_4} \mathcal{M}(\alpha_{ij}, \text{U}, \omega, t) \\
& - \frac{1}{2} \sum_{\langle \langle i,j \rangle \rangle, \sigma_1, \sigma_2, \sigma_3, \sigma_4}\hat{c}_{i \sigma_1}^\dagger \hat{c}_{j \sigma_2} \hat{c}_{j \sigma_3}^\dagger \hat{c}_{i \sigma_4} t_{ij}^{\sigma_1 \sigma_2} t_{ji}^{\sigma_3 \sigma_4} \mathcal{M}(\alpha_{ij}, \text{U}, \omega, t) \\
&= \hat{H}_{\text{eff}}'(t) + \hat{H}_{\text{eff}}''(t)
\end{align*}

$\hat{H}_{\text{eff}}'(t)$ is the same effective Hamiltonian as in the previous section with $\Delta_{ij} = i\Delta_R (R_{ij}^y, - R_{ij}^x, 0)$ and it will lead to the same spin Hamiltonian after introducing spin operators $\hat{H}_{\text{eff}}'(t) = \sum_{\langle i,j \rangle} (J_{ij}(t) \bs{S}_i\bs{S}_j + \bs{D}_{ij}(t)\bs{S}_i \times \bs{S}_j)$ with the same exchange and DMI interaction as in \ref{JijHSOI} and \ref{DijHSOI}. Now we will find the spin Hamiltonian for $\hat{H}_{\text{eff}}''(t)$. Using $\nu_{ij}\nu_{ji} = -1$:

\begin{align*}
\hat{H}_{\text{eff}}''(t) &= -\frac{\Delta^2}{2} \sum_{\langle \langle i,j \rangle \rangle} \left\{ \sum_{\sigma_1 \sigma_2 \sigma_3 \sigma_4} \sigma_{\sigma_1 \sigma_2}^z\sigma_{\sigma_3 \sigma_4}^z \hat{c}_{i\sigma_1}^\dagger\hat{c}_{j\sigma_2}\hat{c}_{j\sigma_3}^\dagger\hat{c}_{i\sigma_4} \right\} \mathcal{M}(\alpha_{ij}, \text{U}, \omega, t)
\end{align*}

For the spin sum we use \ref{SpinOperatorInv1} and \ref{SpinOperatorInv2} taking into account that in the $d=0$ subspace $\hat{n}_{i\uparrow}+\hat{n}_{i\downarrow} = 1$:

\begin{align*}
&\sum_{\sigma_1 \sigma_2 \sigma_3 \sigma_4} \sigma_{\sigma_1 \sigma_2}^z\sigma_{\sigma_3 \sigma_4}^z \hat{c}_{i\sigma_1}^\dagger\hat{c}_{j\sigma_2}\hat{c}_{j\sigma_3}^\dagger\hat{c}_{i\sigma_4} = \sum_{\sigma \sigma'} \sigma_{\sigma \sigma}^z\sigma_{\sigma' \sigma'}^z \hat{c}_{i\sigma}^\dagger\hat{c}_{j\sigma}\hat{c}_{j\sigma'}^\dagger\hat{c}_{i\sigma'} \\
&= \sum_{\sigma \sigma'} \sigma_{\sigma \sigma}^z\sigma_{\sigma' \sigma'}^z \left( \frac{\delta_{\sigma\sigma'}}{2} + \bs{S}_i\bs{\sigma}_{\sigma'\sigma}\right)\left( \frac{\delta_{\sigma\sigma'}}{2} - \bs{S}_j\bs{\sigma}_{\sigma\sigma'}\right) \\
&= \left( \frac{1}{2}+S_i^z \right) \left( \frac{1}{2}-S_j^z \right) - S_i^-(-S_j^+)-S_i^+(-S_j^-) + \left( \frac{1}{2}-S_i^z \right) \left( \frac{1}{2}+S_j^z \right) \\
&= 1-2S_i^zS_j^z+2(S_i^xS_j^x+S_i^yS_j^y)
\end{align*}

Neglecting the constant term, the effective next-nearest neighbors interaction reads:

\begin{equation}
\hat{H}_{\text{eff}}''(t) = \sum_{\langle \langle i,j \rangle \rangle} \tilde{J}_{ij}(t) (S_i^zS_j^z - S_i^xS_j^x - S_i^yS_j^y)
\end{equation}

With:

\begin{equation}
\tilde{J}_{ij}(t) = \Delta^2 \mathcal{M}(\alpha_{ij}, \text{U}, \omega, t)
\end{equation}

This describes a type of anisotropic exchange interaction known as XXZ Heisenberg model for next nearest neighbors. The same spin model is obtain in \cite{Rachel2010} without the laser perturbation. If the laser field is not too strong $\tilde{J}_{ij}(t) > 0$, and, in the $\hat{e}_z$ direction this interaction will compete against the exchange interaction $J_{ij}(t)$. The reason for this is that since $J_{ij}(t) > 0$ the spins will try to attain antiferromagnetic order, which means that next nearest neighbor spin will tend to be aligned. The effect of $\tilde{J}_{ij}(t)$ will be the opposite, forcing the next nearest neighbor spins to be in opposite directions. In the $\hat{e}_x-\hat{e}_y$ plane, $\tilde{J}_{ij}(t)$ will favor ferromagnetic order between next nearest neighbors, which adds to the effect of $J_{ij}(t)$. In general the strength of the exchange interaction will be larger and the net effect of $\tilde{J}_{ij}(t)$ will be a tilting of the spins towards the $\hat{e}_x$-$\hat{e}_y$ plane.

Notice that although the factor $\mathcal{M}(\alpha_{ij}, \text{U}, \omega, t)$ that renormalizes the interaction coupling in presence of the electric field has the same form as in \ref{JijHSOI} and \ref{DijHSOI}, the renormalization will not be the same, since the structure terms $\alpha_{ij}$ will differ from those in a nearest neighbor interaction. For example, if we take the time independent approximation \ref{MFactorApprox}, and use $\lambda = \pm 1$ for circular polarized light, then the exchange interaction will be:

\begin{align*}
J_{ij} = J_{ij}^0 + t_0^2 \mathcal{E}^2 \left( \frac{1}{\text{U}+\omega} + \frac{1}{\text{U}-\omega} \right) = J_{ij}^0 + J_{ij}^0 \text{U}\frac{1}{2} \mathcal{E}^2 \left( \frac{1}{\text{U}+\omega} + \frac{1}{\text{U}-\omega} \right)
\end{align*}

Where $\mathcal{E} = \frac{eaE_0}{\omega}$. And the DMI coupling will be approximated by:

\begin{align*}
\bs{D}_{ij} = \bs{D}_{ij}^0 + 2it_0\bs{\Delta}_{ij}  \mathcal{E}^2 \left( \frac{1}{\text{U}+\omega} + \frac{1}{\text{U}-\omega} \right) = \bs{D}_{ij}^0 + \bs{D}_{ij}^0 \text{U}\frac{1}{2} \mathcal{E}^2 \left( \frac{1}{\text{U}+\omega} + \frac{1}{\text{U}-\omega} \right)
\end{align*}

$J_{ij}^0$ and $\bs{D}_{ij}^0$ have been defined in \ref{Jij0} and \ref{Dij0}, the factor $\frac{1}{2}$ is due to the light being circularly polarized. In contrast, the anisotropic exchange coupling will be:

\begin{align*}
\tilde{J}_{ij} = \tilde{J}_{ij}^0 + 3 \Delta^2 \mathcal{E}^2 \left( \frac{1}{\text{U}+\omega} + \frac{1}{\text{U}-\omega} \right) = \tilde{J}_{ij}^0 + \tilde{J}_{ij}^0 \text{U} \frac{3}{2} \mathcal{E}^2 \left( \frac{1}{\text{U}+\omega} + \frac{1}{\text{U}-\omega} \right)
\end{align*}

Where $\tilde{J}_{ij}^0 = \frac{2\Delta^2}{\text{U}}$ is the anisotropic exchange coupling in the absence of electromagnetic field, and the factor $3$ arises from the structure factor $\alpha_{ij}$ being $\sqrt{3}$ times larger for next nearest neighbors in a honeycomb lattice. If we define $\Delta J_{ij} = J_{ij} - J_{ij}^0$ and $\Delta \tilde{J}_{ij} = \tilde{J}_{ij} - \tilde{J}_{ij}^0$ we can see that, in this approximation $\frac{\Delta J_{ij}}{\Delta \tilde{J}_{ij}} = \frac{1}{3}\frac{J_{ij}^0}{\tilde{J}_{ij}^0}$ whereas $\frac{\Delta J_{ij}}{\Delta |\bs{D}_{ij}|} = \frac{J_{ij}^0}{|\bs{D}_{ij}^0|}$.

Altogether, the total effective Hamiltonian in this system is:

\begin{align}
\hat{H}_{\text{eff}}(t) = &\hat{H}_{\text{eff}}'(t) + \hat{H}_{\text{eff}}''(t) = \sum_{\langle i,j \rangle} \left( J_{ij}(t)\bs{S}_i\bs{S}_j + \bs{D}_{ij}(t) \bs{S}_i \times \bs{S}_j \right) \nonumber \\
&+ \sum_{\langle \langle i,j \rangle \rangle} \tilde{J}_{ij}(t) (S_i^zS_j^z - S_i^xS_j^x - S_i^yS_j^y)
\end{align}

Where $J_{ij}(t)$ and $\bs{D}_{ij}(t)$ have been obtained in \ref{JijHSOI} and \ref{DijHSOI} and we take $\Delta_{ij} = i\Delta_R (R_{ij}^y, - R_{ij}^x, 0)$ to obtain:

\begin{align*}
J_{ij}(t) &= t_0^2\mathcal{M}(\alpha_{ij}, \text{U}, \omega, t) \\
\bs{D}_{ij}(t) &= -2t_0\Delta_R (R_{ij}^y, - R_{ij}^x, 0) \mathcal{M}(\alpha_{ij}, \text{U}, \omega, t) \\
\tilde{J}_{ij}(t) &= \Delta^2 \mathcal{M}(\alpha_{ij}, \text{U}, \omega, t)
\end{align*}

\end{section}
