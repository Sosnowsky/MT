\chapter{Chapter Title Here} % Main chapter title

\begin{section}{Free particle in a periodic potential}

The one particle part of the Hamiltonian can be written as:

\begin{equation}
\hat{H}^{(1)}(\textbf{r}) = -\frac{\hbar}{2m}\Delta^2_{\textbf{r}} + U(\textbf{r})
\end{equation}

Where $U(\textbf{r})$ has the symmetry of the lattice, i.e. $U(\textbf{r} + \textbf{R}) = U(\textbf{r})$ for all Bravais lattice vectors $\textbf{R}$(\ref{AP1A}). The group of translation symmetries $T_{\textbf{R}}\phi(\textbf{r}) = \phi(\textbf{r} + \textbf{R})$ is an abelian group, therefore its irreducible representations are one-dimensional, that is, within an energy level $n$ the wavefunctions of that level transform according to the representation they belong: $T_{\textbf{R}}\phi_{n, \textbf{k}}(\textbf{r}) = \Gamma^{\textbf{k}}(\textbf{R}) \phi_{n, \textbf{k}}(\textbf{r})$. Now, imposing periodic boundary conditions, $\phi_{n, \textbf{k}}(\textbf{r} + L_\mu \textbf{a}_\mu) = \Gamma^{\textbf{k}}(L_\mu \textbf{a}_\mu) \phi_{n, \textbf{k}}(\textbf{r}) = \Gamma^{\textbf{k}}(\textbf{a}_\mu)^{L_\mu} \phi_{n, \textbf{k}}(\textbf{r}) = \phi_{n, \textbf{k}}(\textbf{r})$, so that, $\Gamma^{\textbf{k}}(\textbf{a}_\mu)^{L_\mu} = 1$, this can be accomplished if we label the representation with a vector of the first Brillouin zone and have $\Gamma^{\textbf{k}}(R) = e^{i \textbf{k} \textbf{R}}$. Therefore, for a wavefunction in a periodic lattice we have:

\begin{equation}
\label{Bloch1}
\phi_{n, \textbf{k}}(\textbf{r}+\textbf{R}) = e^{i\textbf{k}\textbf{R}}\phi_{n, \textbf{k}} (\textbf{r})
\end{equation}

Which is the well-known Bloch function form. Here, $n$ is called the band index, and $\textbf{k}$ is the quasimomentum. The energy of this wavefunction is $\epsilon_{n \textbf{k}}$.  Notice that both $\phi_{n,\textbf{k}}$ and $\epsilon_{n \textbf{k}}$ are periodic functions of $\textbf{k}$ in the reciprocal lattice.

The Bloch function is extended over the whole crystal volume $V$. We would like to work with a localized basis. An alternative orthonormal basis are the Wannier functions, defined in terms of the Bloch functions as:

\begin{equation}
\psi_{in}(\textbf{r}) = \frac{1}{\sqrt{M}}\sum_{\textbf{k}\in BZ} e^{-i\textbf{k}\textbf{r}_i} \psi_{n\textbf{k}}(\textbf{r})
\end{equation}

Where $M$ is the number of lattice sites as defined in \ref{AP1A}. Using \ref{Bloch1} we see that $\psi_{in}(\textbf{r}) = \psi_{0n}(\textbf{r}-\textbf{r}_i) \equiv \psi_{n}(\textbf{r}-\textbf{r}_i)$, so we only need to define one Wannier function for each band and the others are obtained by translations.

\end{section}

\begin{section}{Derivation of the single-band Hubbard model}

The Hamiltonian for a system of $N_e$ electrons has the form:

\begin{equation}
\hat{H} = \sum_{i=1}^{N_e} \hat{H}^{(1)}(\textbf{r}_i) + \frac{1}{2} \sum_{\substack{i,j = 1 \\ i \neq j}} ^ {N_e} v(\textbf{r}_i - \textbf{r}_j)
\end{equation}

Where $v$ represents the electron-electron interaction.

\end{section}


\begin{subappendices}
\begin{section}{Crystal structure and reciprocal lattice}
\label{AP1A}

Here we will introduce some notation. The Bravais lattice in a crystal is the lattice generated by the primitive translations $\textbf{a}_\mu$:

\begin{equation}
\textbf{R} = \sum_{\mu=1}^3 m_\mu \textbf{a}_\mu
\end{equation}

Where $m_\mu$ are integers. The volume of the unit cell is $v = \textbf{a}_1 \cdot (\textbf{a}_2 \times \textbf{a}_3)$. The crystal volume is $V = Mv$ where $M = L_1 L_2 L_3$ is the number of lattice sites. The primitive translations in the reciprocal lattice are defined as $\textbf{b}_1 = \frac{2 \pi}{v}\textbf{a}_2 \times \textbf{a}_3$, etc. With this notation the first Brillouin zone is:

\begin{equation}
\textbf{k} = \sum_{\mu=1}^3 \kappa_\mu \textbf{b}_\mu
\end{equation}

Where $\kappa_\mu = \frac{\nu_\mu}{L_\mu}, -\frac{L_\mu}{2}+1 \leq \nu_\mu \leq \frac{L_\mu}{2}$ so that, $-\frac{1}{2} < \kappa_\mu \leq \frac{1}{2}$. We will only consider cubic lattices where $|\textbf{a}_\mu| = a$ and $|b_\mu| = \frac{2\pi}{a}$ and the vectors of the first Brillouin zone have components $k_\mu = \frac{2 \pi \nu_\mu}{aL}$.

\end{section}
\end{subappendices}
