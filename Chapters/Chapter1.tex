\chapter{Chapter Title Here} % Main chapter title

\begin{section}{Free particle in a periodic potential}

The one particle part of the Hamiltonian can be written as:

\begin{equation}
\hat{H}^{(1)}(\textbf{r}) = -\frac{\hbar}{2m}\Delta^2_{\textbf{r}} + U(\textbf{r})
\end{equation}

Where $U(\textbf{r})$ has the symmetry of the lattice, i.e. $U(\textbf{r} + \textbf{R}) = U(\textbf{r})$ for all Bravais lattice vectors $\textbf{R}$(\ref{AP1A}). The group of translation symmetries $T_{\textbf{R}}\phi(\textbf{r}) = \phi(\textbf{r} + \textbf{R})$ is an abelian group, therefore its irreducible representations are one-dimensional, that is, within an energy level $n$ the wavefunctions of that level transform according to the representation they belong: $T_{\textbf{R}}\phi_{n, \textbf{k}}(\textbf{r}) = \Gamma^{\textbf{k}}(\textbf{R}) \phi_{n, \textbf{k}}(\textbf{r})$. Now, imposing periodic boundary conditions, $\phi_{n, \textbf{k}}(\textbf{r} + L_\mu \textbf{a}_\mu) = \Gamma^{\textbf{k}}(L_\mu \textbf{a}_\mu) \phi_{n, \textbf{k}}(\textbf{r}) = \Gamma^{\textbf{k}}(\textbf{a}_\mu)^{L_\mu} \phi_{n, \textbf{k}}(\textbf{r}) = \phi_{n, \textbf{k}}(\textbf{r})$, so that, $\Gamma^{\textbf{k}}(\textbf{a}_\mu)^{L_\mu} = 1$, this can be accomplished if we label the representation with a vector of the first Brillouin zone and have $\Gamma^{\textbf{k}}(R) = e^{i \textbf{k} \textbf{R}}$. Therefore, for a wavefunction in a periodic lattice we have:

\begin{equation}
\label{Bloch1}
\phi_{n, \textbf{k}}(\textbf{r}+\textbf{R}) = e^{i\textbf{k}\textbf{R}}\phi_{n, \textbf{k}} (\textbf{r})
\end{equation}

Which is the well-known Bloch function form. Here, $n$ is called the band index, and $\textbf{k}$ is the quasimomentum. The energy of this wavefunction is $\epsilon_{n \textbf{k}}$.  Notice that both $\phi_{n,\textbf{k}}$ and $\epsilon_{n \textbf{k}}$ are periodic functions of $\textbf{k}$ in the reciprocal lattice.

The Bloch function is extended over the whole crystal volume $V$. We would like to work with a localized basis. An alternative orthonormal basis are the Wannier functions, defined in terms of the Bloch functions as:

\begin{equation}
\psi_{in}(\textbf{r}) = \frac{1}{\sqrt{M}}\sum_{\textbf{k}\in BZ} e^{-i\textbf{k}\textbf{r}_i} \psi_{n\textbf{k}}(\textbf{r})
\end{equation}

Where $M$ is the number of lattice sites as defined in \ref{AP1A}. Using \ref{Bloch1} we see that $\psi_{in}(\textbf{r}) = \psi_{0n}(\textbf{r}-\textbf{r}_i) \equiv \psi_{n}(\textbf{r}-\textbf{r}_i)$, so we only need to define one Wannier function for each band and the others are obtained by translations. 

\end{section}

\begin{section}{Derivation of the single-band Hubbard model}

The Hamiltonian for a system of $N_e$ electrons has the form:

\begin{equation}
\label{Ham1}
\hat{H} = \sum_{i=1}^{N_e} \hat{H}^{(1)}(\textbf{r}_i) + \frac{1}{2} \sum_{\substack{i,j = 1 \\ i \neq j}} ^ {N_e} v(\textbf{r}_i - \textbf{r}_j)
\end{equation}

Where $v$ represents the electron-electron interaction and $\hat{H}^{(1)}$ is the free particle Hamiltonian. We use the Wannier functions $\Phi_j$ as a basis and restrict ourselves to a fixed band so we don't write the band index. Let $\hat{c}_{j \sigma}^\dagger$ and $\hat{c}_{j \sigma}$ create and annihilate a particle in the state $\Phi_j$ and spin $\sigma$. In this basis, the second quantized form of \ref{Ham1} is:

\begin{equation}
\hat{H} = -\sum_{i,j,\sigma} t_{i,j} \hat{c}_{i \sigma}^\dagger \hat{c}_{j \sigma} + \frac{1}{2} \sum_{i,j,k,l, \sigma, \sigma'} \bra{ij} \hat{v} \ket{kl} \hat{c}_{i \sigma}^\dagger \hat{c}_{j \sigma'}^\dagger \hat{c}_{l \sigma'} \hat{c}_{k \sigma}
\end{equation}

Where $t_{ij} = \bra{i} \hat{H}^{(1)} \ket{j} = t_{ji}^*$. And where we assume $\hat{H}^{(1)}$ and $\hat{v}$ to be spin independent. Also we assume that $t_{ii} = t(0) = 0$. For a system of strongly correlated electrons we can assume that:

\begin{itemize}
\item A property of the Wannier functions $\Phi_j$ is that they have exponentially decreasing overlaps, therefore $t_{ij}$ will decay rapidly with the distance $|\textbf{r}_i-\textbf{r}_j|$. In an isotropic system we can approximate:

\begin{equation}
t_{ij} = \begin{cases}
             t,  & \text{for } (i,j) \text{ nearest neighbous} \\
             0,  & \text{otherwise}
       \end{cases} \quad
\end{equation}

\item The dominant contribution of $\bra{ij} \hat{v} \ket{kl}$ will come from $i=j=k=l$ since the Coulomb interaction will be screened for further electrons. Therefore we approximate:

\begin{equation}
\bra{ij} \hat{v} \ket{kl} =
	\begin{cases}
		U, & \text{if } i=j=k=l \\
		0, & \text{otherwise}
	\end{cases}						
\end{equation}

Additionally, for $i=j=k=l$ the Pauli principle requires $\sigma' = \sigma$

\end{itemize}

Taking these approximations we obtain the single band Hubbard model:

\begin{equation}
\label{Hubbard}
\hat{H} = -t\sum_{\langle i,j \rangle, \sigma}(\hat{c}_{i \sigma}^\dagger \hat{c}_{j \sigma} + h.c.) + U \sum_{i=1}^M \hat{n}_{i\uparrow}\hat{n}_{i\downarrow}
\end{equation}

\end{section}

\begin{section}{Heisenberg antiferromagnet from Hubbard model}

At half filling, in the $t << U$ limit we can apply perturbation theory to obtain the Heisenberg model from the Hubbard Hamiltonian \ref{Hubbard}. First notice that in the case $t = 0$ the ground state corresponds to having all sites single-occupied. There are two possible spin orientations for each sites, so there is a $2^M$ degeneracy. The ground state energy is $0$. The first exited states are obtained by moving one electron from one site to another, thus leaving one empty site and one double-occupied site. The energy of one of these states is $U$. Now, if we turn the kinetic term on as a perturbation, having two neighbors electrons with antiparallel spin will allow them to hop and reduce the kinetic energy, whereas while being parallel this hopping is not allowed and there is no energy reduction. Therefore, we can see that when turning the kinetic term on antiparallel alignment will be favored. 

Let us write $\hat{H} = \hat{H_0} + \hat{H_t}$, $\hat{H_t} = -t\sum_{\langle i,j \rangle, \sigma}(\hat{c}_{i \sigma}^\dagger \hat{c}_{j \sigma} + h.c.)$. Let $\ket{\phi}$ and $\ket{\phi'}$ be state from the ground state, then we know $\bra{\phi'} \hat{H_0} \ket{\phi} = 0$ and in first order in $t$ $\bra{\phi'} \hat{H} \ket{\phi} = \bra{\phi'} \hat{H_t} \ket{\phi} = 0$ since $\hat{c}_{i \sigma}^\dagger \hat{c}_{j \sigma}$ moves an electron from site $j$ to $i$ thus leaving $i$ doubly occupied, therefore $\hat{H_t} \ket{\phi}$ has no superposition with $\phi'$. In second order we have:

\begin{equation}
\bra{\phi'} \hat{H} \ket{\phi} = \sum_s \frac{\bra{\phi'} \hat{H_t} \ket{s}\bra{s} \hat{H_t} \ket{\phi}}{E_0-E_s}
\end{equation}

The sum being over all excited states. Since $\hat{H_t}\ket{\phi}$ is a superposition of states with exactly one double occupied site and one empty site, we see that only excited states with one double occupied site will contribute. These states have energy $E_s = U$ therefore we have $\bra{\phi'} \hat{H} \ket{\phi} = -\frac{1}{U}\sum_s \bra{\phi'} \hat{H_t} \ket{s}\bra{s} \hat{H_t} \ket{\phi} = \bra{\phi'} -\frac{\hat{H_t^2}}{U} \ket{\phi}$. These means that within the subspace generated by the ground states of $\hat{H_0}$, $\hat{H}$ acts as an effective Hamiltonian $\hat{H}_{eff} = -\frac{\hat{H_t^2}}{U}$. Now,

\begin{align*}
	\bra{\phi'} -\frac{\hat{H_t^2}}{U} \ket{\phi} =
	\bra{\phi'} -\frac{t^2}{U}\sum_{\langle i,j \rangle, \sigma} \sum_{\langle i',j' \rangle, \sigma'} \hat{c}_{i \sigma}^\dagger 			\hat{c}_{j \sigma} \hat{c}_{i' \sigma'}^\dagger \hat{c}_{j' \sigma'} \ket{\phi} =
	\bra{\phi'} -\frac{t^2}{U}\sum_{\langle i,j \rangle, \sigma, \sigma'} \hat{c}_{i \sigma}^\dagger \hat{c}_{j \sigma} \hat{c}_{j \sigma'}^\dagger \hat{c}_{i \sigma'} \ket{\phi} &= \\
	\bra{\phi'} -\frac{t^2}{U}\sum_{\langle i,j \rangle, \sigma, \sigma'} \hat{c}_{i \sigma}^\dagger \hat{c}_{i \sigma'} \hat{c}_{j \sigma'}^\dagger \hat{c}_{j \sigma'} \ket{\phi} &
\end{align*}

Where we imposed $i=j'$ and $j=i'$ because that's the only way $\hat{H_t^2}\ket{\phi}$ has no double occupied sites and so the matrix element does not vanish. Let us now introduce the on-site spin operators:

\begin{equation}
\boldsymbol{S}_i = \frac{1}{2} \sum_{\sigma, \sigma'} \hat{c}_{i \sigma}^\dagger \boldsymbol{\sigma}_{\sigma, \sigma'} \hat{c}_{i \sigma'}
\end{equation}

Which satisfy:

\begin{align*}
\hat{c}_{i \sigma}^\dagger \hat{c}_{i \sigma'} &= \delta_{\sigma \sigma'} \frac{1}{2} (n_{i \uparrow} + n_{i \downarrow}) + \boldsymbol{S}_i\boldsymbol{\sigma}_{\sigma', \sigma} \\ 
\hat{c}_{i \sigma} \hat{c}_{i \sigma'}^\dagger &= \delta_{\sigma \sigma'} \frac{1}{2} (2 - n_{i \uparrow} - n_{i \downarrow}) - \boldsymbol{S}_i\boldsymbol{\sigma}_{\sigma, \sigma'} 
\end{align*}

Introducing these relations into $H_{eff}$ and using $n_{i \uparrow} + n_{i \downarrow} = 1$ we obtain:

\begin{align*}
H_{eff} = -\frac{t^2}{U} \sum_{\langle i,j \rangle, \sigma \sigma'} \left(\frac{1}{2}\delta_{\sigma \sigma'} + \boldsymbol{S}_i				\boldsymbol{\sigma}_{\sigma' \sigma}\right)\left(\frac{1}{2}\delta_{\sigma \sigma'}-\boldsymbol{S}_j\boldsymbol{\sigma}				_{\sigma \sigma'}\right) = \\
-\frac{t^2}{U}\sum_{\langle i,j \rangle, \sigma \sigma'} \frac{1}{4}\delta_{\sigma \sigma'} - \sum_{ab} S_i^aS_j^b 							\boldsymbol{\sigma}_{\sigma'\sigma}^a \boldsymbol{\sigma}_{\sigma \sigma'}^b = 
-\frac{t^2}{U}\sum_{\langle i,j \rangle} \frac{1}{2} - \sum_{ab} S_i^aS_j^b	2\delta_{ab} &= \\
-\frac{t^2}{U}\sum_{\langle i,j \rangle} \frac{1}{2} - 2\boldsymbol{S}_i\boldsymbol{S}_j
\end{align*}

The constant term can be neglected leaving:

\begin{equation}
H_{eff} = \frac{2t^2}{U} \sum_{\langle i,j \rangle} \boldsymbol{S}_i\boldsymbol{S}_j
\end{equation}

Which is the Heisenberg model for antiferromagnets. 

\end{section}


\begin{subappendices}
\begin{section}{Crystal structure and reciprocal lattice}
\label{AP1A}

Here we will introduce some notation. The Bravais lattice in a crystal is the lattice generated by the primitive translations $\textbf{a}_\mu$:

\begin{equation}
\textbf{R} = \sum_{\mu=1}^3 m_\mu \textbf{a}_\mu
\end{equation}

Where $m_\mu$ are integers. The volume of the unit cell is $v = \textbf{a}_1 \cdot (\textbf{a}_2 \times \textbf{a}_3)$. The crystal volume is $V = Mv$ where $M = L_1 L_2 L_3$ is the number of lattice sites. The primitive translations in the reciprocal lattice are defined as $\textbf{b}_1 = \frac{2 \pi}{v}\textbf{a}_2 \times \textbf{a}_3$, etc. With this notation the first Brillouin zone is:

\begin{equation}
\textbf{k} = \sum_{\mu=1}^3 \kappa_\mu \textbf{b}_\mu
\end{equation}

Where $\kappa_\mu = \frac{\nu_\mu}{L_\mu}, -\frac{L_\mu}{2}+1 \leq \nu_\mu \leq \frac{L_\mu}{2}$ so that, $-\frac{1}{2} < \kappa_\mu \leq \frac{1}{2}$. We will only consider cubic lattices where $|\textbf{a}_\mu| = a$ and $|b_\mu| = \frac{2\pi}{a}$ and the vectors of the first Brillouin zone have components $k_\mu = \frac{2 \pi \nu_\mu}{aL}$.

\end{section}
\end{subappendices}
