\chapter{AFM spinotronics}

\section{History and background}

Antiferromagnetism was first studied by L. N\'eel who introduced a thermodynamic theory following the Weiss's model of a local molecular field. This made it possible to explain the antiferromagnetic order and it predicted the existence of a temperature above which this order vanishes. This temperature is known as N\'eel temperature. Later N\'eel proposed the picture of a magnetic order where two sublattices were magnetised in opposite directions. During the same period, Lev Landau developed a phenomenological model using a combination of two antiferromagnetically coupled ferromagnetic layers.

The discovery of the giant magnetoresistance effect by P. Grünberg \cite{Binasch1989} and A. Fert \cite{Baibich1988} in the 80s opened the gate for a new field of study known as spinotronics, in which electron spin is exploited in improved electronic devices. Magnetoresistive random-access memory (MRAM) is an example of a new type of memory in which data is stored in magnetic elements \citep{Akerman2005}. Newer techniques include thermal assisted switching (TA-MRAM) \citep{Bandiera2015} and spin transfer torque (SPRAM) \citep{Kawahara2012}.

However, so far, in most current spinotronic devices antiferromagnets play only a minor role. During several decades after the discovery of antiferromagnetics, these materials were percived as useless from a practical point of view. On contrast, ferromagnets have been widely studied historically for its technical applications. However, the development of information technology demands devices with high storage density, high energy efficiency and high write-read speeds. Therefore, controlling magnetically ordered systems on subpicosecond timescales is currently a widely studied area. Antiferromagnets aim to complement or even replace ferromagnets due to their rigidity to external magnetic field, and the abscence of stray field.

It has been shown that spin-transfer torques and giant magnetoresistance effects can occur in circuits containing only normal and antiferromagnetic materials \cite{MacDonald2011}. Additionally, spin axis reorientation by a lateral electrical current can be used to offer ultra-fast electrical writing and reading in AFM devices \cite{Zelezny2014}.

Our goal in this project is to investigate the techniques to manipulate the magnetic state in an antiferromagnetic insulator, which could potantially lead to applications for ultra-fast writing devices. For ferromagnetic materials it has already been shown that short laser pulses can induce an effective magnetic field that can reach an amplitude of a few Tesla (\cite{Qaiumzadeh2016}, \cite{Qaiumzadeh2013}).

