\chapter{AFM spinotronics}

\section{History and background}

Antiferromagnetism was first studied by L. Neel who introduced a thermodinamic theory that made it possible to explain the antiferromagnetic order and it pretict the existance of a temperature above which this order vanishes. This temperature is known as Neel temperature. In the next section we will present a semiclassical model for antiferromagnetic materials.

During several decades after the discovery of antiferromagnetics, these materials were percived as useless from a practical point of view.On contrast, ferromagnets have been widely studied historically for its technical applications. However, in the frame of spinotronic devices, the development of information technology demands devices with high storage density, high energy efficiency and high write-read speeds. Therefore, controlling magnetically ordered systems on subpicosecond timescales is currently a widely studied area. Some progress has been done using ferromagnetic materials, for example, it has been shown that circularly polarized femtosecond laser pulses can be used to control the spin dynamics in magnets \cite{Kimel2005}. However, antiferromagnets aim to complement or even replace ferromagnets due to their rigidity to external magnetic field, and the abscence of stray field.

It has been shown that spin-transfer torques and giant magnetoresistance effects can occur in circuits containing only normal and antiferromagnetic materials \cite{MacDonald2011}. Additionally, spin axis reorientation by a lateral electrical current can be used to offer ultra-fast electrical writing and reading in AFM devices \cite{Zelezny2014}.

\section{AFM dynamics}

In this section we will show how a continuum one dimensional model can be obtained from the discrete Heisenberg Hamiltonian in order to describe the AFM order and its dynamics. We write the Heisenberg Hamiltonian as $\hat{H} = J \sum_{\langle i,j \rangle} \bs{S}_i\bs{S}_j - K \sum_i S_{iz}^2$ with $J>0$ and $K$ is the anisotropy energy. Let the lattice be a one dimensional chain with $2N$ sites, such lattice can be described by the repetition of $N$ unit cells consisting of two sites $\alpha$ and $\beta$ for the left and right site respectively. Then the ground state of this Hamiltonian is double degenerated and is obtained by aligning the spins in the $\alpha$ sites in the $\hat{z}$ direction and the $\beta$ spins in the $-\hat{z}$ direction or vice versa. Let us introduce the parameters:

\begin{align}
\bs{m}_i &= \frac{\bs{S}_{\alpha}^i + \bs{S}_{\beta}^i}{2S} \\
\bs{l}_i &= \frac{\bs{S}_{\alpha}^i - \bs{S}_{\beta}^i}{2S}
\end{align}

Where $S$ is the spin angular momentum in units of $\hbar$. Inverting these relations, introducing them into the Hamiltonian and neglecting edge terms we obtain:

\begin{align}
\hat{H} &= JS^2 \sum_i^{N-1} (\bs{m}_i-\bs{l}_i)(\bs{m}_i+\bs{l}_i+\bs{m}_{i+1}+\bs{l}_{i+1}) - KS^2\sum_i^N \left[(\bs{m}_{iz}+\bs{l}_{iz})^2 + (\bs{m}_{iz}-\bs{l}_{iz})^2 \right] \nonumber = \\
& 2JS^2\sum_i^{N} (\bs{m}_i^2-\bs{l}_i^2)+\frac{JS^2}{2}\sum_i^{N-1}\left[(\bs{l}_{i+1}-\bs{l}_i)^2-(\bs{m}_{i+1}-\bs{m}_i)^2 \right] \nonumber \\
& + JS^2\sum_i^{N-1} \left[ \bs{m}_i(\bs{l}_{i+1}-\bs{l}_i) - \bs{l}_i(\bs{m}_{i+1}-\bs{m}_i) \right] - 2KS^2\sum_i^N(m_{iz}^2+l_{iz}^2)
\end{align}