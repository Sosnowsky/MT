\chapter{AFM spinotronics}
Historically ferromagnets have been widely studied as spintronic elements. However, today, the development of information technology demands devices with high storage density, high energy efficiency and high write/read speeds. Therefore, controlling magnetically ordered systems on subpicosecond timescales is currently a widely studied area. It has been shown that circularly polarized femtosecond laser pulses can be used to control the spin dynamics in magnets \cite{Kimel2005}. Antiferromagnetic metals have become an interesting alternative due to their rigidity to external magnetic field, and the abscence of stray field.

Antiferromagnetism materials exhibits no net magnetization since the spins in the material align antiparallel to each other. Historiacally, ferromagnetic materials have been more interesting for practical applications, however recent researches show that antiferromagnetism has promising application in next generation spintronic devices. Controlling magnetically ordered systems on subpicosecond timescales.

https://arxiv.org/ftp/arxiv/papers/1506/1506.07507.pdf

Exploiting both spin and charge of the electron in electronic microdevices has lead to a tremendous progress in both basic
condensed-matter research and microelectronic applications, resulting in the modern field of spintronics. Current spintronics
relies primarily on ferromagnets while antiferromagnets have traditionally played only a supporting role. Recently,
antiferromagnets have been revisited as potential candidates for the key active elements in spintronic devices.

\section{History and background}

\section{AFM dynamics}
