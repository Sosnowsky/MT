\chapter{AFM spinotronics}

\section{History and background}

Antiferromagnetism was first studied by L. N\'eel who introduced a thermodinamic theory that made it possible to explain the antiferromagnetic order and it pretict the existance of a temperature above which this order vanishes. This temperature is known as N\'eel temperature. In the next section we will present a semiclassical model for antiferromagnetic materials.

During several decades after the discovery of antiferromagnetics, these materials were percived as useless from a practical point of view.On contrast, ferromagnets have been widely studied historically for its technical applications. However, in the frame of spinotronic devices, the development of information technology demands devices with high storage density, high energy efficiency and high write-read speeds. Therefore, controlling magnetically ordered systems on subpicosecond timescales is currently a widely studied area. Some progress has been done using ferromagnetic materials, for example, it has been shown that circularly polarized femtosecond laser pulses can be used to control the spin dynamics in magnets \cite{Kimel2005}. However, antiferromagnets aim to complement or even replace ferromagnets due to their rigidity to external magnetic field, and the abscence of stray field.

It has been shown that spin-transfer torques and giant magnetoresistance effects can occur in circuits containing only normal and antiferromagnetic materials \cite{MacDonald2011}. Additionally, spin axis reorientation by a lateral electrical current can be used to offer ultra-fast electrical writing and reading in AFM devices \cite{Zelezny2014}.

