\chapter{Basic models for solid state}

Many properties of solids can be understood using one-electron models, that is, non-interacting models. When the interaction between the ions and the valence electrons can be mostly neglected one can adapt the free electron approximation. This model is mostly limited to the study of metals and despite its simplicity it successfully explains phenomena related with electrical conductivity and heat capacity. In the opposite limit, we can consider systems in which the ion potential is so large that the electrons are bound in the cores with occasional jumps from site to site. This is known as tight binding model and it can be used to study a wide variety of solids. It is typically used for calculations of the electronic band structure.

Despite the success of the single particle models, there are several phenomena which cannot be explained neglecting electron correlations. For example, high temperature superconductivity cannot be understood with these models. The Hubbard model is the simplest model including electron correlations and despite its simplicity it is not fully understood yet. In this Chapter we will outline the main features of the tight binding model and the Hubbard model, which are basic tools for any condensed matter researcher.

\section{Tight binding}