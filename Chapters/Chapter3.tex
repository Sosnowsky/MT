\chapter{Perturbing the lattice}

In this chapter we will derive an effective Hamiltonian for different types of Hubbard-like models in the presence of a high frequency electromagnetic perturbation. The goal is to manipulate $J$ and $\bs{D}_{ij}$ in \ref{HeffNOEMF} separately. This has already been done for the case of Mott insulators (\cite{Mentink2015}, \cite{Kitamura2017}), in which only the exchange interaction $J$ plays a role.We reproduce the result, and extend the analysis to different systems with spin orbit interaction, thereby modulating $\bs{D}_{ij}$ or other spin interactions. 

\begin{section}{Time dependent effective Hamiltonian}

First we will start we a general Hamiltonian of the type:

\begin{equation}
\hat{H} = -\hat{T} + \text{U}\hat{D}
\end{equation}

Where $\hat{D} = \sum_{i=1}^M \hat{n}_{i\uparrow}\hat{n}_{i\downarrow}$ is the doublon number operator and $\hat{T}$ is any hopping operator. The only assumption we will make about $\hat{T}$ is that it is proportional to any power of $\hat{p}$ in which case we can apply the Peierls substitution (\ref{AP3A}) to introduce to take into account the effect of the electromagnetic perturbation in the lattice. The strength of the on-site interaction $\text{U}$ is much larger than the hopping amplitude, therefore, in a half filling system the zero double occupancies subspace $d=0$ can be taken as the low energy subspace in which the effective Hamiltonian will act.

In terms of on-site creation and annihilation operators we can write the hopping operator as $\hat{T} = \sum_{i,j, \sigma, \sigma'} t_{ij}^{\sigma \sigma'} \hat{c}_{i \sigma}^\dagger \hat{c}_{j \sigma'}$. In the simplest case, for example, the sum would be restricted to nearest neighbors and the hopping amplitude would be diagonal in the spin state. With this notation, in presence of a vector potential $\vec{A}(t)$ (which we assume to not vary noticeably in the scale of the lattice) the Peierls substitution leads to an extra time dependent phase in the hopping amplitude:

\begin{equation}
t_{ij}^{\sigma \sigma'}(t) = t_{ij}^{\sigma \sigma'} e^{ie\bs{R}_{ij}\bs{A}(t)}
\end{equation}

Where $\bs{A}$ is the vector potential and $\bs{R}_{ij} = \bs{R}_i-\bs{R}_j$. We write the electric field as $\bs{E}(t) = \frac{1}{2}(\vec{E}e^{-i\omega t}+\vec{E}^*e^{i\omega t})$, where $\vec{E} = E_0\hat{e}$ and $\hat{e} = \frac{1}{\sqrt{1+\lambda_{POL}^2}}(\hat{e}_x+i\lambda_{POL}\hat{e}_y)$ is the polarization vector and $\lambda_{POL} = 0, \pm 1$ for plane polarized, right handed and left handed circular polarized field respectively. The vector potential takes the form $\bs{A}(t) = \frac{1}{2}(\vec{A}e^{-i\omega t} + \vec{A}^* e^{i\omega t})$, with $\vec{A} = \frac{iE_0}{\omega}\hat{e}$.

Let us define $e\bs{R}_{ij}\vec{A} = \alpha_{ij} e^{i \theta_{ij}}$, with $\alpha_{ij} = \pm|e\bs{R}_{ij}\vec{A}|$ in such a way that:

\begin{align}
\alpha_{ij} &= -\alpha_{ji} \label{alphaSym} \\
\theta_{ij} &= \theta_{ji} \label{thetaSym}
\end{align}

and $\theta_{ij} \in \left[0,\pi\right)$. Then we can apply the Jacobi–Anger expansion:

\begin{align*}
t_{ij}^{\sigma \sigma'}(t) = t_{ij}^{\sigma \sigma'}e^{ie\bs{R}_{ij}\bs{A}(t)} = t_{ij}^{\sigma \sigma'}e^{i\alpha_{ij} \cos(\omega t - \theta_{ij})} = t_{ij}^{\sigma \sigma'}\sum_m e^{i(\frac{\pi}{2}-\theta_{ij})m} \mathcal{J}_m(\alpha_{ij}) e^{im\omega t} = \sum t_{ij,m}^{\sigma \sigma'} e^{im\omega t}
\end{align*}

Where we defined 

\begin{equation}
\label{HoppAmpFourier}
t_{ij,m}^{\sigma \sigma'} = t_{ij}^{\sigma \sigma'} e^{i(\frac{\pi}{2}-\theta_{ij})m} \mathcal{J}_m(\alpha_{ij})
\end{equation}

Which is the \textit{m}th Fourier mode of the hopping term and $\mathcal{J}_m(x)$ is the \textit{m}th Bessel function \cite{Kitamura2017}. Correspondingly we can write $\hat{T}(t) = \sum_m \hat{T}_m e^{im \omega t}$ where $\hat{T}_m$ is the sum of all the \textit{m}th Fourier mode of the hopping terms. We can further decompose the hopping operator into:

\begin{equation}
\hat{T}(t) = \sum_m (\hat{T}_{-1,m}+\hat{T}_{0,m}+\hat{T}_{1,m})e^{im\omega t}
\end{equation}

Where $\hat{T}_{dm}(t)$ changes the doublon number by $d$, for example, if $\hat{P}_d$ is the projection operator into the subspace with doublon number $d$, then $\hat{T}_{dm}(t) = \sum_i \hat{P}_{i+d}\hat{T}_{m}(t)\hat{P}_i$.

In order to derive the form of the effective Hamiltonian let us introduce a time dependent unitary transformation $\hat{U}(t) = e^{-i\hat{S}(t)}$. The transformed Hamiltonian is:

\begin{equation}
\hat{H}'(t) = e^{i\hat{S}(t)} \hat{H}(t) e^{-i\hat{S}(t)} - e^{i\hat{S}(t)} id_t e^{-i\hat{S}(t)}
\end{equation} 

We perform the unitary transformation perturbatively in the hopping operator, we can formally write $\hat{T}(t) = \eta \hat{T}(t)$, where $\eta$ will play the role of a bookkeeping parameter in the perturbative expansion. We expand $\hat{S}(t) = \sum_\nu \eta^\nu \hat{S}^{(\nu)}(t)$ and $\hat{H}'(t) = \sum_\nu \eta^\nu \hat{H}'^{(\nu)}(t)$. In order for the new Hamiltonian to be periodic we impose the unitary transformation to have the periodicity of the Hamiltonian, so that we can expand $\hat{S}^{(\nu)}(t) = \sum_m e^{im\omega t}\hat{S}^{(\nu)}_m$. Additionally we can impose the transformed Hamiltonian to be block diagonal in the doublon number $d$. With these conditions the unitary transformation can be uniquely determined if we impose that $\hat{S}(t)$ does not contain block-diagonal terms, i.e. we can write:

\begin{equation}
\hat{S}^{(\nu)}(t) = \sum_{d \neq 0} \sum_m \eta^\nu \hat{S}^{(\nu)}_{d,m} e^{im\omega t}
\end{equation}

where $\hat{S}^{(\nu)}_{d,m}$ changes the double occupancy number by $d$.

Now we can use the identity:

\begin{equation}
id_t e^{-i\hat{S}(t)} = \sum_n \frac{1}{(n+1)!}\text{ad}_{-i\hat{S}(t)}^n (d_t \hat{S}(t))e^{-i\hat{S}(t)}
\end{equation}

Derived in appendix \ref{AP3B}, where $\text{ad}_A(B) = [A,B]$ to rewrite the transformed Hamiltonian as:

\begin{equation}
\hat{H}'(t) = e^{i\hat{S}(t)} \left( \hat{H}(t) - \sum_n \frac{1}{(n+1)!}\text{ad}_{-i\hat{S}(t)}^n (d_t \hat{S}(t)) \right) e^{-i\hat{S}(t)}
\end{equation}

Using the Baker–Campbell–Hausdorff expansion we can write the transformed Hamiltonian as:

\begin{equation}
\label{PertFull}
\hat{H}'(t) = \sum_m \frac{1}{m!} \text{ad}_{i\hat{S}(t)}^m \left( \hat{H}(t) - \sum_n \frac{1}{(n+1)!}\text{ad}_{-i\hat{S}(t)}^n (d_t \hat{S}(t)) \right)
\end{equation}

Now, in this expression we have to expand $\hat{S}(t) = \sum_\nu \eta^\nu \hat{S}^{(\nu)}(t)$ and $\hat{H}'(t) = \sum_\nu \eta^\nu \hat{H}'^{(\nu)}(t)$ and determine $\hat{S}^{(\nu)}(t)$ iteratively in $\nu$ so that $\hat{H}'^{(\nu)}(t)$ is diagonal in the doublon number. Notice that we do not expand $\hat{H}(t)$ as an infinite series since $\hat{H}(t) = -\eta \hat{T}(t) + \text{U}\hat{D}$. Now, notice also that if we take $i\hat{S}^{(0)}(t)=0$ in \ref{PertFull}, then, to zeroth order in $\eta$ only terms in $m=0$ and $n=0$ contribute obtaining:

\begin{equation}
\hat{H}'^{(0)}(t) = \text{U}\hat{D}
\end{equation}

Which is indeed diagonal in the doublon number, and therefore we can take $i\hat{S}^{(0)}(t)=0$. With this equating terms of the same order in \ref{PertFull} is greatly simplified, since terms with high $m,n$ will correspond to terms with high $\eta$. Since we are only interested in orders up to second order, we can truncate \ref{PertFull} up to $m=2$:

\begin{align}
\label{PertTrunc}
\hat{H}'(t) &=  \hat{H}(t) - \sum_n \frac{1}{(n+1)!}\text{ad}_{-i\hat{S}(t)}^n (d_t \hat{S}(t)) + [i\hat{S}(t), \hat{H}(t) - \sum_n \frac{1}{(n+1)!}\text{ad}_{-i\hat{S}(t)}^n (d_t \hat{S}(t))] + \nonumber \\
&+ \frac{1}{2} [i\hat{S}(t),[i\hat{S}(t), \hat{H}(t) - \sum_n \frac{1}{(n+1)!}\text{ad}_{-i\hat{S}(t)}^n (d_t \hat{S}(t)) ]]
\end{align}

This equation holds up to order $\eta^2$. In first order we obtain:

\begin{equation}
\label{1stO}
\hat{H}'^{(1)}(t) = -\hat{T}(t) - d_t\hat{S}^{(1)}(t) + \left[ i\hat{S}^{(1)}(t), \text{U}\hat{D} \right]
\end{equation}

Now we expanding in $m,d$ and use $\left[ \hat{D}, \hat{S}^{(\nu)}_{dm} \right] = d\hat{S}^{(\nu)}_{dm}$ because $\hat{S}^{(\nu)}_{dm}$ changes the doublon number by $d$:

\begin{equation}
\hat{H}'^{(1)}(t)=-\hat{T}(t)-\sum_{d\neq 0}\sum_m (\text{U}d+m\omega) i\hat{S}^{(1)}_{dm} e^{im\omega t}
\end{equation}

Therefore:

\begin{align}
i\hat{S}^{(1)}_d(t) &= -\sum_m \frac{\hat{T}_{d,m}}{\text{U}d+m\omega}e^{im\omega t} \label{1stOSpin}\\
\hat{H}'^{(1)}(t) &= -\sum_m \hat{T}_{0,m}(t)e^{im\omega t} \label{1stOH}
\end{align}

To second order we find:

\begin{align}
\hat{H}'^{(2)}(t) &= - d_t\hat{S}^{(2)}(t) - \frac{1}{2}\left[-i\hat{S}^{(1)}(t), d_t\hat{S}^{(1)}(t) \right] + \left[i\hat{S}^{(1)}(t), -\hat{T}(t)-\frac{1}{2}d_t\hat{S}^{(1)}(t) \right] +\nonumber \\
&+ \left[i\hat{S}^{(2)}(t), \text{U}\hat{D} \right] + \frac{1}{2} \left[i\hat{S}^{(1)}(t), \left[i\hat{S}^{(1)}(t), \text{U}\hat{D} \right] \right] = \nonumber \\
&= \left[i\hat{S}^{(2)}(t), \text{U} \hat{D} \right] - \left[ i\hat{S}^{(1)}(t), \hat{T}(t) \right] + \frac{1}{2}\left[ i\hat{S}^{(1)}(t), \left[ i\hat{S}^{(1)}(t), \text{U}\hat{D} \right] \right] - d_t\hat{S}^{(2)}(t)
\end{align}

Using \ref{1stO} and using $\left[ i\hat{S}^{(1)}(t), d_t\hat{S}^{(1)}(t)\right] = 0$ due to different Fourier modes of the hopping operator being equal up to a constant (explain better), we can rewrite:

\begin{equation}
\hat{H}'^{(2)}(t) = \left[i\hat{S}^{(2)}(t), \text{U} \hat{D} \right] - \frac{1}{2}\left[ i\hat{S}^{(1)}(t), \hat{T}(t) - \hat{H}'^{(1)}(t)\right] - d_t\hat{S}^{(2)}(t)
\end{equation}

Using \ref{1stOSpin} and \ref{1stOH}, the middle term is:

\begin{align*}
&\left[ i\hat{S}^{(1)}(t), \hat{T}(t) - \hat{H}'^{(1)}(t)\right] = -\left[\sum_m \left( \frac{\hat{T}_{1m}}{\text{U}+m\omega} - \frac{\hat{T}_{-1m}}{\text{U}-m\omega} \right)e^{im \omega t}, \sum_n \left( \hat{T}_{-1n} + 2\hat{T}_{0n} + \hat{T}_{1n} \right) e^{in\omega t} \right] \\
&= -\sum_{mn} \left\{ \frac{2\left[\hat{T}_{1m}, \hat{T}_{0n} \right]}{\text{U}+m\omega} + \frac{\left[\hat{T}_{1m}, \hat{T}_{-1n} \right]}{\text{U}+m\omega} - \frac{\left[\hat{T}_{-1m}, \hat{T}_{1n} \right]}{\text{U}-m\omega} - \frac{2\left[\hat{T}_{-1m}, \hat{T}_{0n} \right]}{\text{U}-m\omega} \right\} e^{i(m+n)\omega t} \\
&= -\sum_{mn} \left\{ \frac{2\left[\hat{T}_{1n}, \hat{T}_{0(m-n)} \right]}{\text{U}+n\omega} + \frac{\left[\hat{T}_{1n}, \hat{T}_{-1(m-n)} \right]}{\text{U}+n\omega} - \frac{\left[\hat{T}_{-1n}, \hat{T}_{1(m-n)} \right]}{\text{U}-n\omega} - \frac{2\left[\hat{T}_{-1n}, \hat{T}_{0(m-n)} \right]}{\text{U}-n\omega} \right\} e^{im\omega t}
\end{align*}

Altogether:

\begin{align*}
\hat{H}'^{(2)}(t) &= \sum_{mn} \left\{ \frac{\left[\hat{T}_{1n}, \hat{T}_{0(m-n)} \right]}{\text{U}+n\omega} + \frac{\left[\hat{T}_{1n}, \hat{T}_{-1(m-n)} \right]}{2(\text{U}+n\omega)} - \frac{\left[\hat{T}_{-1n}, \hat{T}_{1(m-n)} \right]}{2(\text{U}-n\omega)} - \frac{\left[\hat{T}_{-1n}, \hat{T}_{0(m-n)} \right]}{\text{U}-n\omega} \right\} e^{im\omega t} \\
&-\sum_{d\neq 0}\sum_m (\text{U}d+m\omega) i\hat{S}^{(2)}_{dm} e^{im\omega t}
\end{align*}

We obtain:

\begin{align}
i\hat{S}^{(2)}_{dm} &= \sum_n \frac{\left[ \hat{T}_{dn}, \hat{T}_{0(m-n)} \right]}{(\text{U}d+n\omega)(\text{U}d+m\omega)} \label{2ndOSpin}\\
\hat{H}'^{(2)}(t) &= \frac{1}{2}\sum_{mn} \left( \frac{\left[\hat{T}_{1n}, \hat{T}_{-1(m-n)} \right]}{\text{U}+n\omega} - \frac{\left[\hat{T}_{-1n}, \hat{T}_{1(m-n)} \right]}{\text{U}-n\omega} \right) e^{im\omega t} \label{2ndOH}
\end{align}

Now the effective Hamiltonian acts on the subspace of $\hat{P}_0$, therefore we are only interested on $\hat{P}_0 \hat{H}'^{(2)}(t) \hat{P}_0$ (notice that $\hat{P}_0 \hat{H}'^{(1)}(t) \hat{P}_0 = 0$). In this subspace we can write:

\begin{align}
\hat{H}_{\text{eff}}(t) &= \hat{P}_0\hat{H}'^{(2)}(t)\hat{P}_0 = -\frac{1}{2}\sum_{mn} \left( \frac{\hat{P}_0  \hat{T}_{-1(m-n)}\hat{T}_{1n}\hat{P}_0}{\text{U}+n\omega} + \frac{\hat{P}_0 \hat{T}_{-1n} \hat{T}_{1(m-n)} \hat{P}_0}{\text{U}-n\omega} \right) e^{im\omega t} \nonumber \\
&= -\frac{1}{2}\sum_{mn} \left\{ \frac{\hat{P}_0  (\hat{T}_{-1(m-n)}\hat{T}_{1n} + \hat{T}_{-1-n}\hat{T}_{1(m+n)})\hat{P}_0}{\text{U}+n\omega} \right\} e^{im\omega t}
\end{align}

Now, since the hopping operators act on the subspace $d=0$ and must remain within that subspace, therefore $\hat{P}_0 \hat{T}_{-1a} \hat{T}_{1b} \hat{P}_0$ will be a sum of all possible hoppings between two different sites:

\begin{align*}
\hat{P}_0 \hat{T}_{-1a} \hat{T}_{1b} \hat{P}_0 = \sum_{i,j, \sigma_1, \sigma_2, \sigma_3, \sigma_4} t_{ij,a}^{\sigma_1 \sigma_2} t_{ji,b}^{\sigma_3 \sigma_4} \hat{c}_{i \sigma_1}^\dagger \hat{c}_{j \sigma_2} \hat{c}_{j \sigma_3}^\dagger \hat{c}_{i \sigma_4}
\end{align*}

Where $a$ and $b$ are any two Fourier modes. Inserting this into \ref{HeffSimplified} and using the definition of $t_{ij,m}^{\sigma \sigma'} = t_{ij}^{\sigma \sigma'} e^{i(\frac{\pi}{2}-\theta_{ij})m} \mathcal{J}_m(\alpha_{ij})$ and using that $\theta_{ji} = \theta_{ij}$ and $\alpha_{ji} = -\alpha_{ij}$ and the properties $\mathcal{J}_{-m}(x) = (-1)^m\mathcal{J}_m(x)$ and $\mathcal{J}_m(-x) = (-1)^m\mathcal{J}_m(x)$ we can write \ref{HeffSimplified} as:

\begin{align}
&\hat{H}_{\text{eff}}(t) = \hat{P}_0\hat{H}'^{(2)}(t)\hat{P}_0 = - \frac{1}{2}\sum_{mn} \sum_{i,j, \sigma_1, \sigma_2, \sigma_3, \sigma_4}\hat{c}_{i \sigma_1}^\dagger \hat{c}_{j \sigma_2} \hat{c}_{j \sigma_3}^\dagger \hat{c}_{i \sigma_4} \frac{t_{ij,m-n}^{\sigma_1 \sigma_2} t_{ji,n}^{\sigma_3 \sigma_4} + t_{ij,-n}^{\sigma_1 \sigma_2} t_{ji,m+n}^{\sigma_3 \sigma_4}}{\text{U}+n\omega} e^{im\omega t} \nonumber \\
&= - \frac{1}{2}\sum_{mn} \sum_{i,j, \sigma_1, \sigma_2, \sigma_3, \sigma_4}\hat{c}_{i \sigma_1}^\dagger \hat{c}_{j \sigma_2} \hat{c}_{j \sigma_3}^\dagger \hat{c}_{i \sigma_4} t_{ij}^{\sigma_1 \sigma_2} t_{ji}^{\sigma_3 \sigma_4} e^{i(\frac{\pi}{2}-\theta_{ij})m} \frac{ \mathcal{J}_{m-n}(\alpha_{ij}) \mathcal{J}_{n}(\alpha_{ji}) + \mathcal{J}_{-n}(\alpha_{ij}) \mathcal{J}_{m+n}(\alpha_{ji})}{\text{U}+n\omega} e^{im\omega t} \nonumber \\
&= - \frac{1}{2}\sum_{mn} \sum_{i,j, \sigma_1, \sigma_2, \sigma_3, \sigma_4}\hat{c}_{i \sigma_1}^\dagger \hat{c}_{j \sigma_2} \hat{c}_{j \sigma_3}^\dagger \hat{c}_{i \sigma_4} t_{ij}^{\sigma_1 \sigma_2} t_{ji}^{\sigma_3 \sigma_4} e^{i(\frac{\pi}{2}-\theta_{ij})m} (-1)^m \frac{ \mathcal{J}_{n-m}(\alpha_{ij}) \mathcal{J}_{n}(\alpha_{ij}) + \mathcal{J}_{n}(\alpha_{ij}) \mathcal{J}_{m+n}(\alpha_{ij})}{\text{U}+n\omega} e^{im\omega t} \nonumber \\
&= - \frac{1}{2} \sum_{i,j, \sigma_1, \sigma_2, \sigma_3, \sigma_4}\hat{c}_{i \sigma_1}^\dagger \hat{c}_{j \sigma_2} \hat{c}_{j \sigma_3}^\dagger \hat{c}_{i \sigma_4} t_{ij}^{\sigma_1 \sigma_2} t_{ji}^{\sigma_3 \sigma_4} \mathcal{M}(\alpha_{ij}, \text{U}, \omega, t) \label{HeffSimplified}
\end{align}

Where we defined:

\begin{equation}
\mathcal{M}(\alpha_{ij}, \text{U}, \omega, t) = \sum_{mn}e^{-i(\frac{\pi}{2}+\theta_{ij})m} \left\{ 
    \frac{\mathcal{J}_{n-m}(\alpha_{ij})\mathcal{J}_{n}(\alpha_{ij}) + \mathcal{J}_{n}(\alpha_{ij})\mathcal{J}_{m+n}(\alpha_{ij})}{\text{U}+n\omega} \right\}e^{im\omega t}
\end{equation}

In the following sections we will apply what we have derived here for different models of the hopping operator.

\end{section}

\begin{section}{Hubbard model without SOI}

We start by considering the simplest hopping operator, as in the Hubbard model:

\begin{equation}
\hat{H} = -t_0\sum_{\langle i,j \rangle, \sigma} \hat{c}_{i \sigma}^\dagger \hat{c}_{j \sigma} + \text{U} \sum_{i=1}^M \hat{n}_{i\uparrow}\hat{n}_{i\downarrow}
\end{equation}

Which is the Hamiltonian \ref{Hubbard} without DMI interaction. We see that the hopping amplitudes introduced in the previous section take the form $t_{ij}^{\sigma \sigma'} = \delta_{\sigma \sigma'} t_0$ for $i,j$ being nearest neighbors, and $t_{ij}^{\sigma \sigma'} = 0$ otherwise. With this we can directly apply \ref{HeffSimplified} to obtain:

\begin{equation}
\hat{H}_{\text{eff}}(t) = -\frac{t_0^2}{2} \sum_{\langle i,j \rangle, \sigma, \sigma'} \hat{c}_{i \sigma}^\dagger \hat{c}_{j \sigma} \hat{c}_{j \sigma'}^\dagger \hat{c}_{i \sigma'} \mathcal{M}(\alpha_{ij}, \text{U}, \omega, t)
\end{equation}

Now, introducing the spin operators \ref{SpinOperatorInv1} and \ref{SpinOperatorInv2} and summing over the spin states as in \ref{SpinRel1}:

\begin{align*}
\sum_{\sigma, \sigma'} \hat{c}_{i \sigma}^\dagger \hat{c}_{j \sigma} \hat{c}_{j \sigma'}^\dagger \hat{c}_{i \sigma'} = \sum_{\sigma, \sigma'} \left( \frac{\delta_{\sigma \sigma'}}{2} + \bs{S}_i\bs{\sigma}_{\sigma' \sigma} \right) \left( \frac{\delta_{\sigma \sigma'}}{2} - \bs{S}_j\bs{\sigma}_{\sigma \sigma'} \right) = -2\bs{S}_i \bs{S}_j
\end{align*}

Where we neglected the constant term. The effective spin Hamiltonian is thus:

\begin{equation}
\hat{H}_{\text{eff}}(t) = \sum_{\langle i,j \rangle} J_{ij}(t) \bs{S}_i \bs{S}_j
\end{equation}

With 

\begin{align}
J_{ij}(t) &= t_0^2 \mathcal{M}(\alpha_{ij}, \text{U}, \omega, t) \nonumber \\
&=\sum_{mn} t_0^2 e^{-i(\frac{\pi}{2}+\theta_{ij})m}\left(\frac{\mathcal{J}_{n-m}(\alpha_{ij})\mathcal{J}_{n}(\alpha_{ij})+\mathcal{J}_{n}(\alpha_{ij})\mathcal{J}_{n+m}(\alpha_{ij})}{\text{U}+n\omega} \right) e^{im\omega t} \label{Jij1}
\end{align}

After time average this reduces to the $m=0$ term:

\begin{equation}
\mathcal{M}(\alpha_{ij}, \text{U}, \omega, t) \approx \sum_{n} 2 \frac{\mathcal{J}_n(\alpha_{ij})^2}{\text{U}+n\omega}
\end{equation}

For $\omega>>U$ we can truncate this to the three smallest values of $n$, i.e. $n=0, \pm 1$. We can also use that $\alpha_{ij} << 1$ because $\alpha_{ij}$ is proportional to $\vec{A}$ which is proportional to $\omega^{-1}$. Therefore we can use $J_n(x) \approx x^n \text{ for } n>0 \text{ and } x << 1$ to obtain:

\begin{equation}
\label{MFactorApprox}
\mathcal{M}(\alpha_{ij}, \text{U}, \omega, t) \approx 2 \left(\frac{\alpha_{ij}^2}{\text{U}+\omega} +\frac{1}{\text{U}} +\frac{\alpha_{ij}^2}{\text{U}-\omega} \right)
\end{equation}

And the exchange interaction coupling becomes:

\begin{equation}
\label{Jij2}
J_{ij} \approx J_{ij}^0 + 2t_0^2 \alpha_{ij}^2 \left( \frac{1}{\text{U}+\omega} + \frac{1}{\text{U}-\omega} \right)
\end{equation}

The lattice structure is contained in $\alpha_{ij} = \pm|e\bs{R}_{ij}\vec{A}|$. For a square lattices aligned with the coordinate system so that $\vec{a}_1=\hat{e}_x$ and $\vec{a}_2=\hat{e}_y$, we will have $\bs{R}_{ij} = \pm a\hat{e}_x,\pm a\hat{e}_y$, where $a$ is the lattice constant. Using $\vec{A}=\frac{iE_0}{\omega\sqrt{1+\lambda^2}}(\hat{e}_x+i\lambda\hat{e}_y)$ we have:

\begin{equation}
\alpha_{ij} = \begin{cases}
             \pm \frac{eaE_0}{\omega \sqrt{1+\lambda^2}} = \pm \frac{\mathcal{E}}{\sqrt{1+\lambda^2}},  & \text{for } \bs{R}_{ij} = \pm \hat{e}_x \\
             \pm \lambda\frac{eaE_0}{\omega \sqrt{1+\lambda^2}} = \pm \frac{\lambda \mathcal{E}}{\sqrt{1+\lambda^2}},  & \text{for } \bs{R}_{ij} = \pm \hat{e}_y
       \end{cases} \quad
\end{equation}

Where $\mathcal{E} = \frac{eaE_0}{\omega}$. 

For plane polarized light ($\lambda = 0$) the exchange interaction only changes in the direction of the polarization, that is:

\begin{equation}
J_{ij}^{PP} = \begin{cases}
		J_{ij}^0 + 2t_0^2 \mathcal{E}^2 \left( \frac{1}{\text{U}+\omega} + \frac{1}{\text{U}-\omega} \right) & \text{for } \bs{R}_{ij} = \pm \hat{e}_x \\
J_{ij}^0 & \text{for } \bs{R}_{ij} = \pm \hat{e}_y
\end{cases} \quad 
\end{equation}

For circular polarized light ($\lambda=\pm1$) in this approximation the exchange interaction changes in the same way in all the directions:

\begin{equation}
\label{JijCPSQUARE}
J_{ij}^{CP} = J_{ij}^0 + t_0^2 \mathcal{E}^2 \left( \frac{1}{\text{U}+\omega} + \frac{1}{\text{U}-\omega} \right)
\end{equation}

\begin{subsection}{Honeycomb lattice}

Now consider a honeycomb lattice structure, so that the $\alpha_{ij}$ values are given by $\alpha_{ij} = \pm|e\bs{R}_{ij}\vec{A}|$ where $\bs{R}_{ij}$ are the displacement vectors in a honeycomb lattice. There are six such vectors:

\begin{align}
\bs{R}_1^\pm &= \pm a\hat{e}_x \\
\bs{R}_2^\pm &= \pm a\frac{1}{2}(\hat{e}_x + \sqrt{3}\hat{e}_y) \\
\bs{R}_3^\pm &= \pm a\frac{1}{2}(\hat{e}_x - \sqrt{3}\hat{e}_y)
\end{align}

These six possible directions will lead to $\alpha_a^\pm = \pm |e\bs{R}_a\vec{A}|$ where $\vec{A}=\frac{iE_0}{\omega\sqrt{1+\lambda^2}}(\hat{e}_x+i\lambda\hat{e}_y)$:

\begin{align}
\alpha_1^\pm &= \pm \frac{eaE_0}{\omega\sqrt{1+\lambda^2}} \\
\alpha_2^\pm &= \alpha_3^\pm = \pm \frac{eaE_0}{2\omega} \sqrt{\frac{1+3\lambda^2}{1+\lambda^2}} = \pm\frac{eaE_0}{2\omega}\sqrt{1+\frac{2\lambda^2}{1+\lambda^2}}
\end{align}

From here we can see that if we take the same approximation as in \ref{Jij2} we will obtain the same form of exchange interaction for circular polarized light ($\lambda=\pm1$) \ref{JijCPSQUARE} with possibly a different lattice constant $a$. 

\end{subsection}

\end{section}

\begin{section}{Hubbard model with SOI}

Now, let's investigate the effect of the electric field for the Hubbard model with SOI, i.e. \ref{Hubbard}, in this case the hopping operator gets an extra spin-dependent term:

\begin{equation}
\hat{T} = \sum_{\langle i,j \rangle, \sigma, \sigma'}(\delta_{\sigma, \sigma'} t_0 + \bs{\Delta}_{ij} \bs{\sigma}_{\sigma, \sigma'})\hat{c}_{i \sigma}^\dagger \hat{c}_{j \sigma'}
\end{equation}

The vector $\bs{\Delta}_{ij}$ can describe Rashba or Dresselhaus SOI, $\bs{\Delta}_{ij} = i\Delta_R(R_{ij}^y, -R_{ij}^x, 0)$ for Rashba SOI and $\bs{\Delta}_{ij} = i\Delta_R(R_{ij}^x, -R_{ij}^y, 0)$ for Dresselhaus SOI. 
In this case the hopping amplitudes are:

\begin{equation}
\label{HoppHubbSOI}
t_{ij}^{\sigma \sigma'} = \begin{cases}
	(\delta_{\sigma \sigma'}t_0+\bs{\Delta}_{ij} \bs{\sigma}_{\sigma, \sigma'}) & \text{for } i, j \text{ nearest neighbors} \\
	0 & \text{ otherwise}
\end{cases} \quad
\end{equation}
 
And as before $\hat{H} = -\hat{T} + \text{U}\hat{D}$. In the presence of an electromagnetic field we can apply the Peierls substitution as we did before, and the effective Hamiltonian will be given by \ref{HeffSimplified}.  inserting the spin operators \ref{SpinOperatorInv1} and \ref{SpinOperatorInv2} we get:

\begin{align}
\hat{H}_{\text{eff}}(t) &= - \frac{1}{2} \sum_{\langle i,j \rangle, \sigma_1, \sigma_2, \sigma_3, \sigma_4}\hat{c}_{i \sigma_1}^\dagger \hat{c}_{j \sigma_2} \hat{c}_{j \sigma_3}^\dagger \hat{c}_{i \sigma_4} t_{ij}^{\sigma_1 \sigma_2} t_{ji}^{\sigma_3 \sigma_4} \mathcal{M}(\alpha_{ij}, \text{U}, \omega, t) \nonumber \\
&= - \frac{1}{2} \sum_{\langle i,j \rangle, \sigma_1, \sigma_2, \sigma_3, \sigma_4} \left( \frac{\delta_{\sigma_1 \sigma_4}}{2} + \bs{S}_i\bs{\sigma}_{\sigma_4 \sigma_1} \right) \left( \frac{\delta_{\sigma_2 \sigma_3}}{2} - \bs{S}_j\bs{\sigma}_{\sigma_2 \sigma_3} \right) \nonumber \\ &(\delta_{\sigma_1 \sigma_2}t_0+\bs{\Delta}_{ij} \bs{\sigma}_{\sigma_1, \sigma_2}) (\delta_{\sigma_3 \sigma_4}t_0+\bs{\Delta}_{ji} \bs{\sigma}_{\sigma_3, \sigma_4}) \mathcal{M}(\alpha_{ij}, \text{U}, \omega, t) \label{HeffHubbSOI1}
\end{align}

Now, since the spin orbit interaction is weaker than the kinetic term, we can neglect the term proportional to $\bs{\Delta}_{ij}^2$ and write:

\begin{equation}
(\delta_{\sigma_1 \sigma_2}t_0+\bs{\Delta}_{ij} \bs{\sigma}_{\sigma_1, \sigma_2}) (\delta_{\sigma_3 \sigma_4}t_0+\bs{\Delta}_{ji} \bs{\sigma}_{\sigma_3, \sigma_4}) \approx \delta_{\sigma_1 \sigma_2}\delta_{\sigma_3 \sigma_4}t_0^2 + t_0\bs{\Delta}_{ij}(\delta_{\sigma_3 \sigma_4} \bs{\sigma}_{\sigma_1 \sigma_2} - \delta_{\sigma_1 \sigma_2} \bs{\sigma}_{\sigma_3 \sigma_4})
\end{equation}

Where we used $\bs{\Delta}_{ji} = - \bs{\Delta}_{ij}$. The term proportional to $t_0^2$ will lead to the modified exchange interaction, and it is exactly the same as in the previous section. The term proportional to $t_0\bs{\Delta}_{ij}$ will lead to the modified DMI interaction.

Now we can rewrite \ref{HeffHubbSOI1} as:

\begin{align*}
\hat{H}_{\text{eff}}(t) &\approx - \frac{1}{2} \sum_{\langle i,j \rangle, \sigma_1, \sigma_2, \sigma_3, \sigma_4} \left( \frac{\delta_{\sigma_1 \sigma_4}}{2} + \bs{S}_i\bs{\sigma}_{\sigma_4 \sigma_1} \right) \left( \frac{\delta_{\sigma_2 \sigma_3}}{2} - \bs{S}_j\bs{\sigma}_{\sigma_2 \sigma_3} \right) \nonumber \\ &\left[ \delta_{\sigma_1 \sigma_2}\delta_{\sigma_3 \sigma_4}t_0^2 + t_0\bs{\Delta}_{ij}(\delta_{\sigma_3 \sigma_4} \bs{\sigma}_{\sigma_1 \sigma_2} - \delta_{\sigma_1 \sigma_2} \bs{\sigma}_{\sigma_3 \sigma_4}) \right] \mathcal{M}(\alpha_{ij}, \text{U}, \omega, t) \\
&= -\frac{t_0^2}{2} \sum_{\langle i,j \rangle \sigma, \sigma'} \left(\frac{1}{2}\delta_{\sigma \sigma'} + \bs{S}_i\bs{\sigma}_{\sigma' \sigma}\right)\left(\frac{1}{2}\delta_{\sigma \sigma'}-\bs{S}_j\bs{\sigma}_{\sigma \sigma'}\right)\mathcal{M}(\alpha_{ij}, \text{U}, \omega, t) - \\
&- \frac{t_0}{2} \sum_{\langle i,j \rangle \sigma_1, \sigma_2, \sigma_3, \sigma_4} \left(\frac{1}{2}\delta_{\sigma_1 \sigma_4} + \bs{S}_i\bs{\sigma}_{\sigma_4 \sigma_1}\right)\left(\frac{1}{2}\delta_{\sigma_2 \sigma_3}-\bs{S}_j\bs{\sigma}_{\sigma_2 \sigma_3}\right) \times \\
&\bs{\Delta}_{ij}(\delta_{\sigma_3,\sigma_4}\bs{\sigma}_{\sigma_1 \sigma_2}-\delta_{\sigma_1,\sigma_2}\bs{\sigma}_{\sigma_3 \sigma_4})\mathcal{M}(\alpha_{ij}, \text{U}, \omega, t) = \\
&= \sum_{\langle i,j \rangle} \left( t_0^2 \bs{S}_i\bs{S}_j + 2it_0\bs{\Delta}_{ij} \bs{S}_i \times \bs{S}_j \right) \mathcal{M}(\alpha_{ij}, \text{U}, \omega, t) =\\
&= \sum_{\langle i,j \rangle} J_{ij}(t)\bs{S}_i\bs{S}_j +\bs{D}_{ij}(t)\bs{S}_i \times \bs{S}_j = \\
&= \hat{H}_J'(t)+\hat{H}_D'(t)
\end{align*}

Where we used relations \ref{SpinRel1} and \ref{SpinRel2}. We have:

\begin{align}
J_{ij}(t) &= t_0^2\mathcal{M}(\alpha_{ij}, \text{U}, \omega, t) \label{JijHSOI} \\
\bs{D}_{ij}(t) &= 2it_0\bs{\Delta}_{ij} \mathcal{M}(\alpha_{ij}, \text{U}, \omega, t) \label{DijHSOI}
\end{align}

Kinetic hopping and the Rashba spin orbit are both NN hopping processes, so they are both renormalized in the same way by the laser field \cite{Stepanov2017}.
For $\bs{D}_{ij}(t)$ this is:

\begin{equation}
\bs{D}_{ij} = 2it_0 \bs{\Delta}_{ij} \sum_{mn}e^{-i(\frac{\pi}{2}+\theta_{ij})m} \left\{ 
    \frac{\mathcal{J}_{n-m}(\alpha_{ij})\mathcal{J}_{n}(\alpha_{ij}) + \mathcal{J}_{n}(\alpha_{ij})\mathcal{J}_{m+n}(\alpha_{ij})}{\text{U}+n\omega} \right\}e^{im\omega t}
\end{equation}

\end{section}

\begin{section}{Kane-Mele-Hubbard model}

The first model to describe topological insulators was introduced by Kane and Mele \cite{Kane2005} to describe quantum spin Hall effect in graphene. In a honeycomb lattice time reversal symmetry and inversion symmetry allow only next-nearest neighbor spin orbit coupling, which is known as intrinsic spin orbit coupling. However, if inversion symmetry is broken by i.e. an external electric field in the $\hat{z}$ direction, a Rashba spin orbit interaction term is allowed between nearest neighbors. In these circumstances the system can be modeled by the Kane-Mele-Hubbard model \cite{Laubach2014}:

\begin{align}
\hat{H}_{KMH} &= -t_0\sum_{\langle i j \rangle \sigma} \hat{c}^{\dagger}_{i\sigma}\hat{c}_{j\sigma} + i\Delta \sum_{\langle \langle i j \rangle \rangle \sigma \sigma'} \hat{c}^{\dagger}_{i\sigma} \nu_{ij} \sigma^z_{\sigma \sigma'} \hat{c}_{j\sigma'} \nonumber \\
&+ i\Delta_R \sum_{\langle i j \rangle \sigma \sigma'} \hat{c}^{\dagger}_{i\sigma} \hat{z}(\bs{\sigma}_{\sigma \sigma'} \times \bs{R}_{ij}) \hat{c}_{j\sigma'} + \text{U}\hat{D}
\end{align}

Where sum over next-nearest neighbors is denoted by $\langle \langle i j \rangle \rangle$ and $\Delta$ is the intrinsic spin orbit coupling and $\Delta_R$ is the Rashba spin orbit coupling, which is the same as in the previous section if we choose $\bs{\Delta}_{ij} = i\Delta_R (R_{ij}^y, - R_{ij}^x, 0)$. $\nu_{ij}=\pm 1$ depending on whether the electron traversing from i to j makes a right ($+1$) or a left turn ($-1$). In this case the hopping amplitudes can be written as:

\begin{equation}
\label{HoppHubbSOI}
t_{ij}^{\sigma \sigma'} = \begin{cases}
	\delta_{\sigma \sigma'}t_0- i\Delta_R \hat{z}(\bs{\sigma}_{\sigma \sigma'} \times \bs{R}_{ij}) & \text{for } i, j \text{ nearest neighbors} \\
	-i\Delta \nu_{ij} \sigma_{\sigma \sigma'}^z & \text{for } i, j \text{ next-nearest neighbors} \\
	0 & \text{ otherwise}
\end{cases} \quad
\end{equation}

We can apply \ref{HeffSimplified} with this hopping amplitudes to obtain:

\begin{align*}
\hat{H}_{\text{eff}}(t) &= - \frac{1}{2} \sum_{\langle i,j \rangle, \sigma_1, \sigma_2, \sigma_3, \sigma_4}\hat{c}_{i \sigma_1}^\dagger \hat{c}_{j \sigma_2} \hat{c}_{j \sigma_3}^\dagger \hat{c}_{i \sigma_4} t_{ij}^{\sigma_1 \sigma_2} t_{ji}^{\sigma_3 \sigma_4} \mathcal{M}(\alpha_{ij}, \text{U}, \omega, t) \\
& - \frac{1}{2} \sum_{\langle \langle i,j \rangle \rangle, \sigma_1, \sigma_2, \sigma_3, \sigma_4}\hat{c}_{i \sigma_1}^\dagger \hat{c}_{j \sigma_2} \hat{c}_{j \sigma_3}^\dagger \hat{c}_{i \sigma_4} t_{ij}^{\sigma_1 \sigma_2} t_{ji}^{\sigma_3 \sigma_4} \mathcal{M}(\alpha_{ij}, \text{U}, \omega, t) \\
&= \hat{H}_{\text{eff}}'(t) + \hat{H}_{\text{eff}}''(t)
\end{align*}

$\hat{H}_{\text{eff}}'(t)$ is the same effective Hamiltonian as in the previous section with $\Delta_{ij} = i\Delta_R (R_{ij}^y, - R_{ij}^x, 0)$ and it will lead to the same spin Hamiltonian after introducing spin operators $\hat{H}_{\text{eff}}'(t) = \sum_{\langle i,j \rangle} (J_{ij}(t) \bs{S}_i\bs{S}_j + \bs{D}_{ij}(t)\bs{S}_i \times \bs{S}_j)$ with the same exchange and DMI interaction as in \ref{JijHSOI} and \ref{DijHSOI}. Now we will find the spin Hamiltonian for $\hat{H}_{\text{eff}}''(t)$. Using $\nu_{ij}\nu_{ji} = -1$:

\begin{align*}
\hat{H}_{\text{eff}}''(t) &= -\frac{\Delta^2}{2} \sum_{\langle \langle i,j \rangle \rangle} \left\{ \sum_{\sigma_1 \sigma_2 \sigma_3 \sigma_4} \sigma_{\sigma_1 \sigma_2}^z\sigma_{\sigma_3 \sigma_4}^z \hat{c}_{i\sigma_1}^\dagger\hat{c}_{j\sigma_2}\hat{c}_{j\sigma_3}^\dagger\hat{c}_{i\sigma_4} \right\} \mathcal{M}(\alpha_{ij}, \text{U}, \omega, t)
\end{align*}

For the spin sum we use \ref{SpinOperatorInv1} and \ref{SpinOperatorInv2} taking into account that in the $d=0$ subspace $\hat{n}_{i\uparrow}+\hat{n}_{i\downarrow} = 1$:

\begin{align*}
&\sum_{\sigma_1 \sigma_2 \sigma_3 \sigma_4} \sigma_{\sigma_1 \sigma_2}^z\sigma_{\sigma_3 \sigma_4}^z \hat{c}_{i\sigma_1}^\dagger\hat{c}_{j\sigma_2}\hat{c}_{j\sigma_3}^\dagger\hat{c}_{i\sigma_4} = \sum_{\sigma \sigma'} \sigma_{\sigma \sigma}^z\sigma_{\sigma' \sigma'}^z \hat{c}_{i\sigma}^\dagger\hat{c}_{j\sigma}\hat{c}_{j\sigma'}^\dagger\hat{c}_{i\sigma'} \\
&= \sum_{\sigma \sigma'} \sigma_{\sigma \sigma}^z\sigma_{\sigma' \sigma'}^z \left( \frac{\delta_{\sigma\sigma'}}{2} + \bs{S}_i\bs{\sigma}_{\sigma'\sigma}\right)\left( \frac{\delta_{\sigma\sigma'}}{2} - \bs{S}_j\bs{\sigma}_{\sigma\sigma'}\right) \\
&= \left( \frac{1}{2}+S_i^z \right) \left( \frac{1}{2}-S_j^z \right) - S_i^-(-S_j^+)-S_i^+(-S_j^-) + \left( \frac{1}{2}-S_i^z \right) \left( \frac{1}{2}+S_j^z \right) \\
&= 1-2S_i^zS_j^z+2(S_i^xS_j^x+S_i^yS_j^y)
\end{align*}

Neglecting the constant term, the effective next-nearest neighbors interaction reads:

\begin{equation}
\hat{H}_{\text{eff}}''(t) = \sum_{\langle \langle i,j \rangle \rangle} \bs{S}_i \bs{\Gamma}_{ij}(t) \bs{S}_j 
\end{equation}

With:

\begin{equation}
\bs{\Gamma}_{ij}(t) = \Delta^2 \text{diag}(-1,-1,1) \mathcal{M}(\alpha_{ij}, \text{U}, \omega, t) 
\end{equation}

This is the same pseudodipolar interaction derived in the previous Chapter with no laser interaction. In this case it describes a type of anisotropic exchange interaction known as XXZ Heisenberg model for next nearest neighbors. The same spin model is obtain in \cite{Rachel2010} without the laser perturbation. If the laser field is not too strong so that we can assume $\mathcal{M}(\alpha_{ij}, \text{U}, \omega, t) > 0$, then $\Gamma^{zz}_{ij}(t) > 0$. Therefore we see that this interaction favors antiferromagnetic order in the $\hat{e}_z$ direction and ferromagnetic order in the $\hat{e}_x-\hat{e}_y$ plane. The exchange interaction $J_{ij}(t)$ will favor antiferromagnetic order for nearest neighbors, so that next nearest neighbors will tend to be aligned. Therefore, $\Gamma^{zz}_{ij}(t)$ will compete against $J_{ij}(t)$ in the $\hat{e}_z$ direction. In the $\hat{e}_x-\hat{e}_y$ plane, $\bs{\Gamma}_{ij}(t)$ will favor ferromagnetic order between next nearest neighbors, which adds to the effect of $J_{ij}(t)$. In general the strength of the exchange interaction will be larger and the net effect of $\bs{\Gamma}_{ij}(t)$ will be a tilting of the spins towards the $\hat{e}_x$-$\hat{e}_y$ plane.

Notice that although the factor $\mathcal{M}(\alpha_{ij}, \text{U}, \omega, t)$ that renormalizes the interaction coupling in presence of the electric field has the same form as in \ref{JijHSOI} and \ref{DijHSOI}, the renormalization will not be the same, since the structure terms $\alpha_{ij}$ will differ from those in a nearest neighbor interaction. For example, if we take the time independent approximation \ref{MFactorApprox}, and use $\lambda = \pm 1$ for circular polarized light, then the exchange interaction will be:

\begin{align*}
J_{ij} = J_{ij}^0 + t_0^2 \mathcal{E}^2 \left( \frac{1}{\text{U}+\omega} + \frac{1}{\text{U}-\omega} \right) = J_{ij}^0 + J_{ij}^0 \text{U}\frac{1}{2} \mathcal{E}^2 \left( \frac{1}{\text{U}+\omega} + \frac{1}{\text{U}-\omega} \right)
\end{align*}

Where $\mathcal{E} = \frac{eaE_0}{\omega}$. And the DMI coupling will be approximated by:

\begin{align*}
\bs{D}_{ij} = \bs{D}_{ij}^0 + 2it_0\bs{\Delta}_{ij}  \mathcal{E}^2 \left( \frac{1}{\text{U}+\omega} + \frac{1}{\text{U}-\omega} \right) = \bs{D}_{ij}^0 + \bs{D}_{ij}^0 \text{U}\frac{1}{2} \mathcal{E}^2 \left( \frac{1}{\text{U}+\omega} + \frac{1}{\text{U}-\omega} \right)
\end{align*}

$J_{ij}^0$ and $\bs{D}_{ij}^0$ have been defined in \ref{Jij0} and \ref{Dij0}, the factor $\frac{1}{2}$ is due to the light being circularly polarized. In contrast, the anisotropic exchange coupling will be:

\begin{align*}
\Gamma^{zz}_{ij} = \Gamma^{0,zz}_{ij} + 3 \Delta^2 \mathcal{E}^2 \left( \frac{1}{\text{U}+\omega} + \frac{1}{\text{U}-\omega} \right) = \tilde{J}_{ij}^0 + \tilde{J}_{ij}^0 \text{U} \frac{3}{2} \mathcal{E}^2 \left( \frac{1}{\text{U}+\omega} + \frac{1}{\text{U}-\omega} \right)
\end{align*}

Where $\Gamma^{0,zz}_{ij}(t) = \frac{2\Delta^2}{\text{U}}$ is the anisotropic exchange coupling in the absence of electromagnetic field, and the factor $3$ arises from the structure factor $\alpha_{ij}$ being $\sqrt{3}$ times larger for next nearest neighbors in a honeycomb lattice. 

Altogether, the total effective Hamiltonian in this system is:

\begin{align}
\hat{H}_{\text{eff}}(t) = &\hat{H}_{\text{eff}}'(t) + \hat{H}_{\text{eff}}''(t) = \sum_{\langle i,j \rangle} \left( J_{ij}(t)\bs{S}_i\bs{S}_j + \bs{D}_{ij}(t) \bs{S}_i \times \bs{S}_j \right) \nonumber \\
&+ \sum_{\langle \langle i,j \rangle \rangle} \bs{S}_i \bs{\Gamma}_{ij}(t) \bs{S}_j 
\end{align}

Where $J_{ij}(t)$ and $\bs{D}_{ij}(t)$ have been obtained in \ref{JijHSOI} and \ref{DijHSOI} and we take $\Delta_{ij} = i\Delta_R (R_{ij}^y, - R_{ij}^x, 0)$ to obtain:

\begin{align*}
J_{ij}(t) &= t_0^2\mathcal{M}(\alpha_{ij}, \text{U}, \omega, t) \\
\bs{D}_{ij}(t) &= -2t_0\Delta_R (R_{ij}^y, - R_{ij}^x, 0) \mathcal{M}(\alpha_{ij}, \text{U}, \omega, t) \\
\bs{\Gamma}_{ij}(t) &= \Delta^2 \text{diag}(-1,-1,1) \mathcal{M}(\alpha_{ij}, \text{U}, \omega, t) 
\end{align*}

\end{section}


\begin{subappendices}
\begin{section}{Peierls Substitution}
\label{AP3A}

In first quantization we can introduce a vector potential $\bs{A}(\bs{r},t)$ by changing the Hamiltonian of the latice \ref{LaticeHam} to:

\begin{equation}
\label{HamEMF}
  \hat{H}'(\bs{r}) = \frac{(\bs{p}-e\bs{A})^2}{2m} +U(\bs{r})
\end{equation}

Now the Bloch functions defined for \ref{LaticeHam} will not be eigenfunctions of this Hamiltonian. We define a new set of Wannier functions in terms of those defined in \ref{Wannier2}, and obtain the new Bloch functions. 

\begin{align}
\psi'_{\bs{R}}(\bs{r}) &= e^{i\frac{e}{\hbar}\int_{\bs{R}}^{\bs{r}} d\bs{r}'\bs{A}(\bs{r}',t)} \psi_{\bs{R}}(\bs{r}) \\
\phi'_{\bs{k}}(\textbf{r}) &= \frac{1}{\sqrt{M}}\sum_{\bs{R}} e^{-i\bs{k}\bs{R}}\psi'_{\bs{R}}(\bs{r}) = \frac{1}{\sqrt{M}}\sum_{\bs{R}} e^{-i\bs{k}\bs{R}} e^{i\frac{e}{\hbar}\int_{\bs{R}}^{\bs{r}} d\bs{r}'\bs{A}(\bs{r}',t)} \psi_{\bs{R}}(\bs{r})
\end{align}

Where we omitted the band number. The action of the new Hamiltonian is greatly simplified:

\begin{align*}
\hat{H}'(\bs{r}) \psi'_{\bs{R}} (\bs{r}) &= \left[ \frac{(\bs{p}-e\bs{A})^2}{2m} +U(\bs{r}) \right] e^{i\frac{e}{\hbar}\int_{\bs{R}}^{\bs{r}} d\bs{r}'\bs{A}(\bs{r}',t)} \psi_{\bs{R}}(\bs{r}) = \\
&=e^{i\frac{e}{\hbar}\int_{\bs{R}}^{\bs{r}} d\bs{r}'\bs{A}(\bs{r}',t)} \left[ \frac{(\bs{p} - i\hbar \bs{\nabla_r}(i\frac{e}{\hbar}\int_{\bs{R}}^{\bs{r}} d\bs{r}'\bs{A}(\bs{r}',t)) -e\bs{A})^2}{2m} +U(\bs{r}) \right] = \\
&= e^{i\frac{e}{\hbar}\int_{\bs{R}}^{\bs{r}} d\bs{r}'\bs{A}(\bs{r}',t)} \left[ \frac{\bs{p}^2}{2m} + U(\bs{r}) \right] \psi_{\bs{R}}(\bs{r}) = e^{i\frac{e}{\hbar}\int_{\bs{R}}^{\bs{r}} d\bs{r}'\bs{A}(\bs{r}',t)} \hat{H}(\bs{r}) \psi_{\bs{R}}(\bs{r})
\end{align*}

And so, the matrix elements as in the case without field except for a phase:

\begin{equation}
\label{PeierlsMtxElem}
t_{ij}(t) = \bra{\psi'_{\bs{R_i}}} \hat{H}'(\bs{r}) \ket{\psi'_{\bs{R_j}}} = e^{i\frac{e}{\hbar}\int_{\bs{R}_j}^{\bs{R}_i} d\bs{r}'\bs{A}(\bs{r}',t)} \bra{\psi_{\bs{R_i}}} \hat{H}(\bs{r}) \ket{\psi_{\bs{R_j}}}
\end{equation}

When the field can be approximated as constant along the lattice we have: $e^{i\frac{e}{\hbar}\int_{\bs{R}_j}^{\bs{R}_i} d\bs{r}'\bs{A}(\bs{r}',t)} = e^{i\frac{e}{\hbar} (\bs{R}_i-\bs{R}_j) \bs{A}(t)}$.

Notice that this substitution also applies if we add a Rashba Spin-Orbit interaction term to \ref{HamEMF}, i.e., if $\hat{H}'(\bs{r}) = \frac{(\bs{p}-e\bs{A})^2}{2m} +U(\bs{r}) + \alpha_R \hat{z}(\hat{\bs{\sigma}} \times (\bs{p}-e\bs{A}))$, then

\begin{align*}
  \alpha_R \hat{z}(\hat{\bs{\sigma}} \times (\bs{p}-e\bs{A})) \psi'_{\bs{R}} &= \alpha_R(\hat{\sigma_x}(\hat{p}_y-e\bs{A}) - \hat{\sigma_y}(\hat{p}_x-e\bs{A})) e^{i\frac{e}{\hbar}\int_{\bs{R}}^{\bs{r}}d\bs{r}'\bs{A}(\bs{r}',t)} \psi_{\bs{R}} = \\
  &= e^{i\frac{e}{\hbar}\int_{\bs{R}}^{\bs{r}}d\bs{r}'\bs{A}(\bs{r}',t)} \alpha_R \hat{z}\hat{\bs{\sigma}} \times \bs{p} \psi_{\bs{R}}
\end{align*}

Therefore relation \ref{PeierlsMtxElem} still holds.

\end{section}

\begin{section}{Identities}
\label{AP3B}
Let $Z(t)$ be an operator dependent on the parameter $t$. Then:

\begin{align*}
d_t e^{Z} &= d_t \sum_{n=0}^\infty \frac{1}{n!} Z^n = \sum_{n=0}^\infty \sum_{m=0}^{n-1} \frac{1}{n!} Z^m \frac{\partial Z}{\partial t} Z^{n-m-1} = \\
&= \sum_{m=0}^\infty \sum_{p=0}^\infty \frac{1}{(m+p+1)!}Z^m \frac{\partial Z}{\partial t} Z^p = \sum_{m=0}^\infty \sum_{p=0}^\infty \frac{m!p!}{(m+p+1)!} \frac{Z^m}{m!} \frac{\partial Z}{\partial t} \frac{Z^p}{p!} =\\
&= \sum_{m=0}^\infty \sum_{p=0}^\infty \int_0^1 dx x^m (1-x)^p \frac{Z^m}{m!} \frac{\partial Z}{\partial t} \frac{Z^p}{p!} = \int_0^1 dx e^{Zx} \frac{\partial Z}{\partial t} e^{Z(1-x)} = \\
&= \int_0^1 dx \sum_{n=0}^\infty \frac{1}{n!} \left[xZ, \dots, \left[xZ, \frac{\partial Z}{\partial t}\right] \dots \right] e^Z = \sum_{n=0}^\infty \frac{1}{(n+1)!} \left[Z, \dots, \left[Z, \frac{\partial Z}{\partial t}\right] \dots \right] e^Z
\end{align*}

Where we used the Beta function $\int_0^1 dx x^m (1-x)^p = \frac{m!p!}{(m+p+1)!}$ and we also used the Baker–Campbell–Hausdorff expansion for $e^{xZ}\frac{\partial Z}{\partial t}e^{-xZ}$ and $\int_0^1 x^n = \frac{1}{n+1}$. Therefore we proved:

\begin{equation}
\label{SneidersID}
d_t e^{Z} = \sum_{n=0}^\infty \frac{1}{(n+1)!} \left[Z, \dots, \left[Z, \frac{\partial Z}{\partial t}\right] \dots \right] e^Z = \sum_{n=0}^\infty \frac{1}{(n+1)!} \text{ad}_Z ^n \left(\frac{\partial Z}{\partial t}\right) e^Z
\end{equation}

Where $\text{ad}_A(B) = [A,B]$.

\end{section}

\end{subappendices}







