\chapter{Perturbing the lattice}

In this chapter we will study how introducing an electromagnetic perturbation in \ref{Ham1} can affect the parameters $J$ and $\boldsymbol{D}_{ij}$ of the effective Hamiltonian. 

\begin{section}{Modified exchange interaction}

We start by considering only the exchange interaction part in a one-dimensional lattice. Consider the Hamiltonian:

\begin{equation}
\hat{H} = -t_0\sum_{\langle i,j \rangle, \sigma} \hat{c}_{i \sigma}^\dagger \hat{c}_{j \sigma} + U \sum_{i=1}^M \hat{n}_{i\uparrow}\hat{n}_{i\downarrow}
\end{equation}

Which is the Hamiltonian \ref{Hubbard} without DMI interaction. We can introduce a time dependent electric field in this Hamiltonian by adding a phase to the hopping term $t_{ij}(t) = t_0e^{ie\bs{R}_{ij}\bs{A}(t)}$, where $\bs{A}$ is the vector potential and $\bs{R}_{ij} = \bs{R}_i-\bs{R}_j$. This is known as Peierls substitution and it is derived in \ref{AP3A}. The electric field has the form $\bs{E}(t) = E_0 \sin (\omega t)\bs{\hat{x}}$ (one-dimensional lattice). We can therefore take $\bs{A}(t) = -\frac{E_0}{\omega}\cos(\omega t)\bs{\hat{x}} = -\bs{A}\cos(\omega t)$. Therefore the time dependent Hamiltonian is:

\begin{equation}
\hat{H}(t) = -\sum_{\langle i,j \rangle, \sigma} t_{ij}(t)\hat{c}_{i \sigma}^\dagger \hat{c}_{j \sigma} + U \sum_{i=1}^M \hat{n}_{i\uparrow}\hat{n}_{i\downarrow} = -\hat{T}(t) + U\hat{D}
\end{equation}

Where $\hat{D}$ is the double occupancy number operator and $\hat{T}(t) = \sum_{\langle i,j \rangle, \sigma} t_{ij}(t)\hat{c}_{i \sigma}^\dagger \hat{c}_{j \sigma}$ is the hopping operator. Since the time dependence in periodic we can expand $t_{ij}(t) = \sum_m t_{ij,m} e^{im \omega t}$, where

\begin{equation}
t_{ij,m} = t_0(-1)^mJ_m(\alpha_{ij})
\end{equation}

Where $\alpha_{ij} = e\bs{R}_{ij}\bs{A}$ and $J_m(x)$ is the \textit{m}th Bessel function, and correspondingly $\hat{T}(t) = \sum_m \hat{T}_m e^{im \omega t}$.

In order to derive the form of the effective Hamiltonian in the half filling system with $\frac{t}{U} << 1$ let us introduce a time dependent unitary transformation $\hat{U}(t) = e^{-i\hat{S}(t)}$. The transformed Hamiltonian is:

\begin{equation}
\hat{H}'(t) = \hat{U}^\dagger (t) \hat{H}(t) \hat{U}(t) - i\hat{U}^\dagger(t) d_t \hat{U}(t)
\end{equation} 

We perform the unitary transformation perturbatively in the hopping parameter $t_0$. We expand $\hat{S}(t) = \sum_\nu \hat{S}^{(\nu)}(t)$ and $\hat{H}'(t) = \sum_\nu \hat{H}'^{(\nu)}(t)$. In order for the new Hamiltonian to be periodic we impose the unitary transformation to have the periodicity of the Hamiltonian, so that we can expand $\hat{S}^{(\nu)}(t) = \sum_m e^{im\omega t}\hat{S}^{(\nu)}_m$. Additionally we can impose the transformed Hamiltonian to be block diagonal in the doublon number $d$. With these conditions the unitary transformation can be uniquely determined if we impose that $\hat{S}(t)$ does not contain block-diagonal terms, i.e. we can write:

\begin{equation}
\hat{S}^{(\nu)}(t) = \sum_{d \neq 0} \sum_m \hat{S}^{(\nu)}_{d,m} e^{im\omega t}
\end{equation}

where $\hat{S}^{(\nu)}_{d,m}$ changes the double occupancy number by $d$. We also decompose the hopping operator as:

\begin{equation}
\hat{T}(t) = \sum_m (\hat{T}_{-1,m}+\hat{T}_{0,m}+\hat{T}_{1,m})e^{im\omega t}
\end{equation}

Where $\hat{T}_d(t)$ changes the double occupancy number by $d$, for example $\hat{T}_1(t) = \hat{P}_1 \hat{T}(t) \hat{P}_0$. We can write the transformed Hamiltonian as:

\begin{equation}
\label{Perturbation}
\hat{H}'(t) = \hat{H}(t) + \sum_{n=1}^\infty \frac{1}{n!} \left[i\hat{S}(t), \dots, \left[ i\hat{S}(t), \hat{H}(t) \right]\dots \right] - d_t \hat{S}(t)
\end{equation}

To zeroth order we have $\hat{H}'^{(0)}= U\hat{D}$ and $i\hat{S}^{(0)}(t)=0$. In first order we obtain:

\begin{equation}
\hat{H}'^{(1)}(t)=-\hat{T}(t)-\sum_{d\neq 0}\sum_m (Ud+m\omega) i\hat{S}^{(1)}_{dm} e^{im\omega t}
\end{equation}

Therefore:

\begin{align*}
i\hat{S}^{(1)}_d(t) &= -\sum_m \frac{\hat{T}_{d,m}}{Ud+m\omega}e^{im\omega t} \\
\hat{H}'^{(1)}(t) &= -\hat{T}_0(t)
\end{align*}

To second order we find:

\begin{equation}
\hat{H}'^{(2)}(t) = \left[i\hat{S}^{(2)}(t), U \hat{D} \right] - \left[ i\hat{S}^{(1)}(t), \hat{T}(t) \right] - d_t\hat{S}^{(2)}(t)
\end{equation}

Where the middle term is:

\begin{align*}
\left[ i\hat{S}^{(1)}(t), \hat{T}(t) \right] &= -\left[\sum_m \left( \frac{\hat{T}_{1m}}{U+m\omega} - \frac{\hat{T}_{-1m}}{U-m\omega} \right)e^{im \omega t}, \sum_n \left( \hat{T}_{-1n} +\hat{T}_{0n} + \hat{T}_{1n} \right) e^{in\omega t} \right] \\
&= -\sum_{mn} \left\{ \frac{\left[\hat{T}_{1m}, \hat{T}_{0n} \right]}{U+m\omega} + \frac{\left[\hat{T}_{1m}, \hat{T}_{-1n} \right]}{U+m\omega} - \frac{\left[\hat{T}_{-1m}, \hat{T}_{1n} \right]}{U-m\omega} - \frac{\left[\hat{T}_{-1m}, \hat{T}_{0n} \right]}{U-m\omega} \right\} e^{i(m+n)\omega t} \\
&= -\sum_{mn} \left\{ \frac{\left[\hat{T}_{1n}, \hat{T}_{0(m-n)} \right]}{U+n\omega} + \frac{\left[\hat{T}_{1n}, \hat{T}_{-1(m-n)} \right]}{U+n\omega} - \frac{\left[\hat{T}_{-1n}, \hat{T}_{1(m-n)} \right]}{U-n\omega} - \frac{\left[\hat{T}_{-1n}, \hat{T}_{0(m-n)} \right]}{U-n\omega} \right\} e^{im\omega t}
\end{align*}

We obtain:

\begin{align*}
i\hat{S}^{(2)}_{dm} &= \sum_n \frac{\left[ \hat{T}_{dn}, \hat{T}_{0(m-n)} \right]}{(Ud+n\omega)(Ud+m\omega)} \\
\hat{H}'^{(2)}(t) &= \sum_{mn} \left( \frac{\left[\hat{T}_{1n}, \hat{T}_{-1(m-n)} \right]}{U+n\omega} - \frac{\left[\hat{T}_{-1n}, \hat{T}_{1(m-n)} \right]}{U-n\omega} \right) e^{im\omega t}
\end{align*}

Now the effective Hamiltonian acts on the subspace of $\hat{P}_0$, therefore we are only interested on $\hat{P}_0 \hat{H}'^{(2)}(t) \hat{P}_0$ (notice that $\hat{P}_0 \hat{H}'^{(1)}(t) \hat{P}_0 = 0$). In this subspace we can write:

\begin{align*}
\hat{T}_{1n} \hat{T}_{-1(m-n)} &= 0 \\
\hat{T}_{-1(m-n)} \hat{T}_{1n} &= \sum_{\langle i,j \rangle, \sigma_1, \sigma_2} t_{ji,(m-n)} t_{ij,n} \hat{c}_{j \sigma_2}^\dagger \hat{c}_{i \sigma_2} \hat{c}_{i \sigma_1}^\dagger \hat{c}_{j \sigma_1}
\end{align*}

So that,

\begin{equation}
\hat{H}_{eff}(t) = \hat{P}_0\hat{H}'^{(2)}(t)\hat{P}_0 = - \sum_{mn} \sum_{\langle i,j \rangle, \sigma_1, \sigma_2}\hat{c}_{j \sigma_2}^\dagger \hat{c}_{i \sigma_2} \hat{c}_{i \sigma_1}^\dagger \hat{c}_{j \sigma_1} \left( \frac{t_{ji,(m-n)} t_{ij,n}}{U+n\omega} + \frac{t_{ji,n} t_{ij,(m-n)}}{U-n\omega} \right) e^{im\omega t}
\end{equation}

Comparing with \ref{PertNoEMF} we can see that this will lead to a spin Hamiltonian 

\begin{equation}
\hat{H}_{eff}(t) = \sum_{\langle i,j \rangle} J_{ij}(t) \bs{S}_i \bs{S}_j
\end{equation}

With 

\begin{align}
J_{ij}(t) &= \sum_{mn} 2\left( \frac{t_{ji,(m-n)} t_{ij,n}}{U+n\omega} + \frac{t_{ji,n} t_{ij,(m-n)}}{U-n\omega} \right) e^{im\omega t} \nonumber \\
&= \sum_{mn} 2(-1)^m t_0^2 \left( \frac{J_{m-n}(\alpha_{ji})J_n(\alpha_{ij})}{U+n\omega} +\frac{J_n(\alpha_{ji})J_{(m-n)}(\alpha_{ij})}{U-n\omega} \right)
\end{align}

In the case $\omega >> U$ we can limit to $m=0$ (time independent part) and $n=0$ or $n=\pm1$ (small denominators), additionally, if the electric field is small we can use $J_n(x) \approx x^n$. With these approximations: 

\begin{equation}
J_{ij} = 4 t_0^2 \left( \frac{\alpha_{ij}^2}{U-\omega} + \frac{1}{U} + \frac{\alpha_{ij}^2}{U+\omega} \right)
\end{equation}

\end{section}

\begin{subappendices}
\begin{section}{Peierls Substitution}
\label{AP3A}

In first quantization we can introduce a vector potential $\bs{A}(\bs{r},t)$ by changing the Hamiltonian of the latice \ref{LaticeHam} to:

\begin{equation}
\label{HamEMF}
  \hat{H}'(\bs{r}) = \frac{(\bs{p}-e\bs{A})^2}{2m} +U(\bs{r})
\end{equation}

Now the Bloch functions defined for \ref{LaticeHam} will not be eigenfunctions of this Hamiltonian. We define a new set of Wannier functions in terms of those defined in \ref{Wannier2}, and obtain the new Bloch functions. 

\begin{align}
\psi'_{\bs{R}}(\bs{r}) &= e^{i\frac{e}{\hbar}\int_{\bs{R}}^{\bs{r}} d\bs{r}'\bs{A}(\bs{r}',t)} \psi_{\bs{R}}(\bs{r}) \\
\phi'_{\bs{k}}(\textbf{r}) &= \frac{1}{\sqrt{M}}\sum_{\bs{R}} e^{-i\bs{k}\bs{R}}\psi'_{\bs{R}}(\bs{r}) = \frac{1}{\sqrt{M}}\sum_{\bs{R}} e^{-i\bs{k}\bs{R}} e^{i\frac{e}{\hbar}\int_{\bs{R}}^{\bs{r}} d\bs{r}'\bs{A}(\bs{r}',t)} \psi_{\bs{R}}(\bs{r})
\end{align}

Where we omitted the band number. The action of the new Hamiltonian is greatly simplified:

\begin{align*}
\hat{H}'(\bs{r}) \psi'_{\bs{R}} (\bs{r}) &= \left[ \frac{(\bs{p}-e\bs{A})^2}{2m} +U(\bs{r}) \right] e^{i\frac{e}{\hbar}\int_{\bs{R}}^{\bs{r}} d\bs{r}'\bs{A}(\bs{r}',t)} \psi_{\bs{R}}(\bs{r}) = \\
&= e^{i\frac{e}{\hbar}\int_{\bs{R}}^{\bs{r}} d\bs{r}'\bs{A}(\bs{r}',t)} \left[ \frac{\bs{p}^2}{2m} + U(\bs{r}) \right] \psi_{\bs{R}}(\bs{r}) = e^{i\frac{e}{\hbar}\int_{\bs{R}}^{\bs{r}} d\bs{r}'\bs{A}(\bs{r}',t)} \hat{H}(\bs{r}) \psi_{\bs{R}}(\bs{r})
\end{align*}

And so, the matrix elements as in the case without field except for a phase:

\begin{equation}
\label{PeierlsMtxElem}
t_{ij}(t) = \bra{\psi'_{\bs{R_i}}} \hat{H}'(\bs{r}) \ket{\psi'_{\bs{R_j}}} = e^{i\frac{e}{\hbar}\int_{\bs{R}_j}^{\bs{R}_i} d\bs{r}'\bs{A}(\bs{r}',t)} \bra{\psi_{\bs{R_i}}} \hat{H}(\bs{r}) \ket{\psi_{\bs{R_j}}}
\end{equation}

When the field can be approximated as constant along the lattice we have: $e^{i\frac{e}{\hbar}\int_{\bs{R}_j}^{\bs{R}_i} d\bs{r}'\bs{A}(\bs{r}',t)} = e^{i\frac{e}{\hbar} (\bs{R}_i-\bs{R}_j) \bs{A}(t)}$.

Notice that this substitution also applies if we add a Rashba Spin-Orbit interaction term to \ref{HamEMF}, i.e., if $\hat{H}'(\bs{r}) = \frac{(\bs{p}-e\bs{A})^2}{2m} +U(\bs{r}) + \alpha_R \hat{z}(\hat{\bs{\sigma}} \times (\bs{p}-e\bs{A}))$, then

\begin{align*}
  \alpha_R \hat{z}(\hat{\bs{\sigma}} \times (\bs{p}-e\bs{A})) \psi'_{\bs{R}} &= \alpha_R(\hat{\sigma_x}(\hat{p}_y-e\bs{A}) - \hat{\sigma_y}(\hat{p}_x-e\bs{A})) e^{i\frac{e}{\hbar}\int_{\bs{R}}^{\bs{r}}d\bs{r}'\bs{A}(\bs{r}',t)} \psi_{\bs{R}} = \\
  &= e^{i\frac{e}{\hbar}\int_{\bs{R}}^{\bs{r}}d\bs{r}'\bs{A}(\bs{r}',t)} \alpha_R \hat{z}\hat{\bs{\sigma}} \times \bs{p} \psi_{\bs{R}}
\end{align*}

Therefore relation \ref{PeierlsMtxElem} still holds.

\end{section}
\end{subappendices}







