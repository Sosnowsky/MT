% !TEX spellcheck = en-US
\documentclass[aps,prl,twocolumn,amsmath,amssymb,superscriptaddress,nobibnotes]{revtex4}%
\usepackage{graphicx}
\usepackage{dcolumn}
\usepackage{bm}
\usepackage{color}
\usepackage{amsmath}
\usepackage{epstopdf}
%\usepackage{caption}
%\usepackage{subcaption}
\usepackage{fancybox}
\usepackage{amsfonts}
\usepackage{amssymb}%
\usepackage[toc,page]{appendix}
%\usepackage[backend=bibtex,natbib=true,sorting=none]{biblatex}
\setcounter{MaxMatrixCols}{30}


\renewcommand{\cite}[1]{{[}\onlinecite{#1}{]}}

\newcommand{\s}{\sum\limits}
\newcommand{\p}{\prod\limits}
\newcommand{\pa}{\partial}
\newcommand{\il}{\int\limits}
\newcommand{\be}{\begin{equation}}
\newcommand{\e}{\end{equation}}
\newcommand{\beml}{\begin{subequations}}
\newcommand{\eml}{\end{subequations}}
\newcommand{\beq}{\begin{eqnarray}}
\newcommand{\eq}{\end{eqnarray}}
\newcommand{\ba}{\begin{array}}
\newcommand{\ea}{\end{array}}
\newcommand{\bpm}{\begin{pmatrix}}
\newcommand{\epm}{\end{pmatrix}}
\newcommand{\bc}{\begin{cases}}
\newcommand{\ec}{\end{cases}}
\newcommand{\lt}{\left}
\newcommand{\rt}{\right}
\newcommand{\n}{\nonumber}
\newcommand{\la}{\langle}
\newcommand{\ra}{\rangle}
\newcommand{\ep}{\varepsilon}
\newcommand{\dd}{\displaystyle}
\newcommand{\bs}{\boldsymbol}
\newcommand{\h}{^\dagger}
\newcommand{\ph}{^{\phantom{\dagger}}}
\newcommand{\one}{\openone}
\newcommand{\ut}{\underaccent{\tilde}}
\newcommand{\ul}{\underaccent{\bar}}
\newcommand{\up}{\uparrow}
\newcommand{\dn}{\downarrow}
\DeclareMathOperator{\var}{var}
\DeclareMathOperator{\tr}{Tr}
\DeclareMathOperator{\diag}{diag}
\DeclareMathOperator{\im}{Im}
\DeclareMathOperator{\re}{Re}
\DeclareMathOperator{\ctg}{cotan}
\DeclareMathOperator{\arcsh}{arcsinh}
\DeclareMathOperator{\csch}{csch}
\DeclareMathOperator{\rot}{rot}

\newcommand{\AQ}[1]{\textbf{\textcolor{red}{{#1}}}}

\begin{document}
\title{Ultrafast manipulation of Heisenberg exchange and Dzyaloshinskii–Moriya interactions in antiferromagnetic insulators}


\begin{abstract}
This is the abstract.
\end{abstract}

\date{\today}
\maketitle

\begin{section}{Model and calculations}

We will start we a general Hamiltonian:

\begin{equation}
\hat{H} = -\sum_{ij\sigma \sigma'} t_{ij}^{\sigma \sigma'} \hat{c}_{i \sigma}^\dagger \hat{c}_{j \sigma'} + \text{U}\hat{D}
\end{equation}

Where $\hat{D} = \sum_{i=1}^M \hat{n}_{i\uparrow}\hat{n}_{i\downarrow}$ is the doublon number operator and $\text{U}$ is the on-site interaction. We will assume a half filling system in which the strength of the on-site interaction $\text{U}$ is much larger than the hopping amplitudes, therefore the zero double occupancies subspace $d=0$ can be taken as the low energy subspace in which the effective Hamiltonian will act.

The hopping term $-\hat{T} = -\sum_{ij\sigma \sigma'} t_{ij}^{\sigma \sigma'} \hat{c}_{i \sigma}^\dagger \hat{c}_{j \sigma'}$ can describe various models. We will be interested in:

\begin{itemize}
	\item The Hubbard model, in which the hopping does not change the spin, $t_{ij}^{\sigma \sigma'} = \delta_{\sigma \sigma'}t$ for $i,j$ NN and $t_{ij}^{\sigma \sigma'} = 0$ otherwise.
	\item The so-called Kane-Mele-Hubbard model, which includes a spin-dependent term araising from spin-obit interactions.
	\begin{equation}
t_{ij}^{\sigma \sigma'} = \begin{cases}
	\delta_{\sigma \sigma'}t_1 & \text{for } i, j \text{ nearest neighbors} \\
	-\Delta \nu_{ij} \hat{e}_z \cdot \bs{\sigma}_{\sigma, \sigma'} & \text{for } i, j \text{ next nearest neighbors} \\
	0 & \text{ otherwise} \n
\end{cases} \quad
\end{equation}
\end{itemize}

In the prescence of an laser perturbation we can use the Peirls substitution to introduce the effect of the field trough the hopping amplitudes. We can write the electric field as $\bs{E}(t) = \frac{1}{2}(\vec{E}e^{-i\omega t}+\vec{E}^*e^{i\omega t})$, $\vec{E} = E_0\hat{e}$ and $\hat{e} = \frac{1}{\sqrt{1+\lambda_{POL}^2}}(\hat{e}_x+i\lambda_{POL}\hat{e}_y)$ is the polarization vector and $\lambda_{POL} = 0, \pm 1$ for plane polarized, right handed and left handed circular polarized field respectively. According to the Peierls rule the hopping amplitudes gain a phase:
\begin{equation}
t_{ij}^{\sigma \sigma'}(t) = t_{ij}^{\sigma \sigma'} e^{ie\bs{R}_{ij}\bs{A}(t)}
\end{equation}
where $\bs{R}_{ij} = \bs{R}_i-\bs{R}_j$, $\bs{R}_i$ is the position of site $i$ and $\bs{A}$ is the vector potential $\bs{A}(t) = \frac{1}{2}(\vec{A}e^{-i\omega t} + \vec{A}^* e^{i\omega t})$, with $\vec{A} = \frac{iE_0}{\omega}\hat{e}$.

Let us define:

\begin{equation}
\label{Def_alpha}
e\bs{R}_{ij}\vec{A} = \alpha_{ij} e^{i \theta_{ij}}
\end{equation}

With $\alpha_{ij} = \pm|e\bs{R}_{ij}\vec{A}|$ in such a way that:

\begin{align}
\alpha_{ij} &= -\alpha_{ji} \label{alphaSym} \\
\theta_{ij} &= \theta_{ji} \label{thetaSym}
\end{align}

and $\theta_{ij} \in \left[0,\pi\right)$. Then we can apply the Jacobi–Anger expansion:

\begin{align*}
t_{ij}^{\sigma \sigma'}(t) = t_{ij}^{\sigma \sigma'}e^{ie\bs{R}_{ij}\bs{A}(t)} = t_{ij}^{\sigma \sigma'}e^{i\alpha_{ij} \cos(\omega t - \theta_{ij})} = t_{ij}^{\sigma \sigma'}\sum_m e^{i(\frac{\pi}{2}-\theta_{ij})m} \mathcal{J}_m(\alpha_{ij}) e^{im\omega t} = \sum t_{ij,m}^{\sigma \sigma'} e^{im\omega t}
\end{align*}

Where we defined 

\begin{equation}
\label{HoppAmpFourier}
t_{ij,m}^{\sigma \sigma'} = t_{ij}^{\sigma \sigma'} e^{i(\frac{\pi}{2}-\theta_{ij})m} \mathcal{J}_m(\alpha_{ij})
\end{equation}

Which is the \textit{m}th Fourier mode of the hopping term and $\mathcal{J}_m(x)$ is the \textit{m}th Bessel function \cite{Kitamura2017}. 
\end{section}

\section*{Acknowledgments}

%\begin{thebibliography}
\bibliography{paper}
%\end{thebibliography}

\end{document}


