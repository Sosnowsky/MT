% !TEX spellcheck = en-US
\documentclass[aps,prl,twocolumn,amsmath,amssymb,superscriptaddress]{revtex4}%
\usepackage{graphicx}
\usepackage{dcolumn}
\usepackage{bm}
\usepackage{color}
\usepackage{amsmath}
\usepackage{epstopdf}
%\usepackage{caption}
%\usepackage{subcaption}
\usepackage{fancybox}
\usepackage{amsfonts}
\usepackage{amssymb}%
\usepackage[toc,page]{appendix}
\setcounter{MaxMatrixCols}{30}


\renewcommand{\cite}[1]{{[}\onlinecite{#1}{]}}

\newcommand{\s}{\sum\limits}
\newcommand{\p}{\prod\limits}
\newcommand{\pa}{\partial}
\newcommand{\il}{\int\limits}
\newcommand{\be}{\begin{equation}}
\newcommand{\e}{\end{equation}}
\newcommand{\beml}{\begin{subequations}}
\newcommand{\eml}{\end{subequations}}
\newcommand{\beq}{\begin{eqnarray}}
\newcommand{\eq}{\end{eqnarray}}
\newcommand{\ba}{\begin{array}}
\newcommand{\ea}{\end{array}}
\newcommand{\bpm}{\begin{pmatrix}}
\newcommand{\epm}{\end{pmatrix}}
\newcommand{\bc}{\begin{cases}}
\newcommand{\ec}{\end{cases}}
\newcommand{\lt}{\left}
\newcommand{\rt}{\right}
\newcommand{\n}{\nonumber}
\newcommand{\la}{\langle}
\newcommand{\ra}{\rangle}
\newcommand{\ep}{\varepsilon}
\newcommand{\dd}{\displaystyle}
\newcommand{\bs}{\boldsymbol}
\newcommand{\h}{^\dagger}
\newcommand{\ph}{^{\phantom{\dagger}}}
\newcommand{\one}{\openone}
\newcommand{\ut}{\underaccent{\tilde}}
\newcommand{\ul}{\underaccent{\bar}}
\newcommand{\up}{\uparrow}
\newcommand{\dn}{\downarrow}
\DeclareMathOperator{\var}{var}
\DeclareMathOperator{\tr}{Tr}
\DeclareMathOperator{\diag}{diag}
\DeclareMathOperator{\im}{Im}
\DeclareMathOperator{\re}{Re}
\DeclareMathOperator{\ctg}{cotan}
\DeclareMathOperator{\arcsh}{arcsinh}
\DeclareMathOperator{\csch}{csch}
\DeclareMathOperator{\rot}{rot}

\newcommand{\AQ}[1]{\textbf{\textcolor{red}{{#1}}}}

\begin{document}
\title{Ultrafast manipulation of Heisenberg exchange and Dzyaloshinskii–Moriya interactions in antiferromagnetic insulators}


\begin{abstract}
This is the abstract.
\end{abstract}

\date{\today}
\maketitle

\begin{section}{Model and calculations}

We will start we a general Hamiltonian:

\begin{equation}
\hat{H} = -\sum_{ij\sigma \sigma'} t_{ij}^{\sigma \sigma'} \hat{c}_{i \sigma}^\dagger \hat{c}_{j \sigma'} + \text{U}\hat{D}
\end{equation}

Where $\hat{D} = \sum_{i=1}^M \hat{n}_{i\uparrow}\hat{n}_{i\downarrow}$ is the doublon number operator and $\text{U}$ is the on-site interaction. The hopping term $-\hat{T} = -\sum_{ij\sigma \sigma'} t_{ij}^{\sigma \sigma'} \hat{c}_{i \sigma}^\dagger \hat{c}_{j \sigma'}$ can describe various models. We will be interested in:

\begin{itemize}
	\item The Hubbard model, in which the hopping does not change the spin, $t_{ij}^{\sigma \sigma'} = \delta_{\sigma \sigma'}t$ for $i,j$ NN and $t_{ij}^{\sigma \sigma'} = 0$ otherwise.
	\item The so-called Kane-Mele-Hubbard model, which includes a spin-dependent term araising from spin-obit interactions.
	\begin{equation}
t_{ij}^{\sigma \sigma'} = \begin{cases}
	\delta_{\sigma \sigma'}t_1 & \text{for } i, j \text{ nearest neighbors} \\
	-\Delta \nu_{ij} \hat{e}_z \cdot \bs{\sigma}_{\sigma, \sigma'} & \text{for } i, j \text{ next nearest neighbors} \\
	0 & \text{ otherwise} \n
\end{cases} \quad
\end{equation}
\end{itemize}

The only assumption we will make about $\hat{T}$ is that it is proportional to any power of $\hat{p}$ in which case we can apply the Peierls substitution (\ref{AP3A}) to introduce to take into account the effect of the electromagnetic perturbation in the lattice. The strength of the on-site interaction $\text{U}$ is much larger than the hopping amplitude, therefore, in a half filling system the zero double occupancies subspace $d=0$ can be taken as the low energy subspace in which the effective Hamiltonian will act.

In terms of on-site creation and annihilation operators we can write the hopping operator as $\hat{T} = \sum_{i,j, \sigma, \sigma'} t_{ij}^{\sigma \sigma'} \hat{c}_{i \sigma}^\dagger \hat{c}_{j \sigma'}$. In the simplest case, for example, the sum would be restricted to nearest neighbors and the hopping amplitude would be diagonal in the spin state. With this notation, in presence of a vector potential $\vec{A}(t)$ (which we assume to not vary noticeably in the scale of the lattice) the Peierls substitution leads to an extra time dependent phase in the hopping amplitude:
\end{section}

\section*{Acknowledgments}

\begin{thebibliography}{9}

\end{thebibliography}

\end{document}


