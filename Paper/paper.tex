% !TEX spellcheck = en-US
\documentclass[aps,prl,twocolumn,amsmath,amssymb,superscriptaddress,nobibnotes]{revtex4}%
\usepackage{graphicx}
\usepackage{dcolumn}
\usepackage{bm}
\usepackage{color}
\usepackage{amsmath}
\usepackage{epstopdf}
%\usepackage{caption}
%\usepackage{subcaption}
\usepackage{fancybox}
\usepackage{amsfonts}
\usepackage{amssymb}%
\usepackage[toc,page]{appendix}
%\usepackage[backend=bibtex,natbib=true,sorting=none]{biblatex}
\setcounter{MaxMatrixCols}{30}


\renewcommand{\cite}[1]{{[}\onlinecite{#1}{]}}

\newcommand{\s}{\sum\limits}
\newcommand{\p}{\prod\limits}
\newcommand{\pa}{\partial}
\newcommand{\il}{\int\limits}
\newcommand{\be}{\begin{equation}}
\newcommand{\e}{\end{equation}}
\newcommand{\beml}{\begin{subequations}}
\newcommand{\eml}{\end{subequations}}
\newcommand{\beq}{\begin{eqnarray}}
\newcommand{\eq}{\end{eqnarray}}
\newcommand{\ba}{\begin{array}}
\newcommand{\ea}{\end{array}}
\newcommand{\bpm}{\begin{pmatrix}}
\newcommand{\epm}{\end{pmatrix}}
\newcommand{\bc}{\begin{cases}}
\newcommand{\ec}{\end{cases}}
\newcommand{\lt}{\left}
\newcommand{\rt}{\right}
\newcommand{\n}{\nonumber}
\newcommand{\la}{\langle}
\newcommand{\ra}{\rangle}
\newcommand{\ep}{\varepsilon}
\newcommand{\dd}{\displaystyle}
\newcommand{\bs}{\boldsymbol}
\newcommand{\h}{^\dagger}
\newcommand{\ph}{^{\phantom{\dagger}}}
\newcommand{\one}{\openone}
\newcommand{\ut}{\underaccent{\tilde}}
\newcommand{\ul}{\underaccent{\bar}}
\newcommand{\up}{\uparrow}
\newcommand{\dn}{\downarrow}
\DeclareMathOperator{\var}{var}
\DeclareMathOperator{\tr}{Tr}
\DeclareMathOperator{\diag}{diag}
\DeclareMathOperator{\im}{Im}
\DeclareMathOperator{\re}{Re}
\DeclareMathOperator{\ctg}{cotan}
\DeclareMathOperator{\arcsh}{arcsinh}
\DeclareMathOperator{\csch}{csch}
\DeclareMathOperator{\rot}{rot}

\newcommand{\AQ}[1]{\textbf{\textcolor{red}{{#1}}}}

\begin{document}
\title{Ultrafast manipulation of Heisenberg exchange and Dzyaloshinskii–Moriya interactions in antiferromagnetic insulators}


\begin{abstract}
This is the abstract.
\end{abstract}

\date{\today}
\maketitle

\begin{section}{Model and calculations}

The first model to describe topological insulators was introduced by Kane and Mele \cite{Kane2005} to describe quantum spin Hall effect in graphene. In a honeycomb lattice time reversal symmetry and inversion symmetry allow only next-nearest neighbor spin orbit coupling, which is known as intrinsic spin orbit coupling. In these circumstances the system can be modeled by the Kane-Mele-Hubbard model:

\begin{align}
\label{MKMH}
\hat{H}_0 &= - \sum_{\langle i,j \rangle, \sigma} t_1\hat{c}_{i \sigma}^\dagger \hat{c}_{j \sigma} + \n \\
	&-\sum_{\langle \langle i,j \rangle \rangle, \sigma}(t_2 - i\Delta\nu_{ij}\sigma^z_{\sigma, \sigma})\hat{c}_{i \sigma}^\dagger \hat{c}_{j \sigma} + 
	\text{U}\hat{D}
\end{align}

Where $\hat{c}_{i \sigma}^\dagger$ ($ \hat{c}_{i \sigma}$) creates (anihilates) an electron at site $i$ in spin state $\sigma$. $\hat{D} = \sum_{i=1}^M \hat{n}_{i\uparrow}\hat{n}_{i\downarrow}$ is the doublon number operator and $\text{U}$ is the on-site interaction. $t_1$, $t_2$ are the hopping amplitudes originating from both kinetic hopping. Where $\Delta$ is the intrinsic spin orbit coupling constant. $\nu_{ij}=\pm 1$ depending on whether the electron traversing from $i$ to $j$ makes a right ($+1$) or a left turn ($-1$). $\sigma^{z}$ is the third Pauli matrix. We will assume a half filling system in which the strength of the on-site interaction $\text{U}$ is much larger than the hopping amplitudes, therefore the zero double occupancies subspace $d=0$ can be taken as the low energy subspace in which the effective Hamiltonian will act.

In the prescence of an laser perturbation we can use the Peirls substitution to introduce the effect of the field trough the hopping amplitudes. We can write the electric field as $\bs{E}(t) = \frac{1}{2}(\vec{E}e^{-i\omega t}+\vec{E}^*e^{i\omega t})$, $\vec{E} = E_0\hat{e}$ and $\hat{e} = \frac{1}{\sqrt{1+\lambda_{POL}^2}}(\hat{e}_x+i\lambda_{POL}\hat{e}_y)$ is the polarization vector and $\lambda_{POL} = 0, \pm 1$ for plane polarized, right handed and left handed circular polarized field respectively. According to the Peierls rule the hopping amplitudes gain a phase $e^{ie\bs{R}_{ij}\bs{A}(t)}$ where $\bs{R}_{ij} = \bs{R}_i-\bs{R}_j$, $\bs{R}_i$ is the position of site $i$ and $\bs{A}$ is the vector potential $\bs{A}(t) = \frac{1}{2}(\vec{A}e^{-i\omega t} + \vec{A}^* e^{i\omega t})$, with $\vec{A} = \frac{iE_0}{\omega}\hat{e}$.

Let us define:

\begin{equation}
\label{Def_alpha}
e\bs{R}_{ij}\vec{A} = \alpha_{ij} e^{i \theta_{ij}}
\end{equation}

With $\alpha_{ij} = \pm|e\bs{R}_{ij}\vec{A}|$ in such a way that:

\begin{align}
\alpha_{ij} &= -\alpha_{ji} \label{alphaSym} \\
\theta_{ij} &= \theta_{ji} \label{thetaSym}
\end{align}

and $\theta_{ij} \in \left[0,\pi\right)$. Then we can apply the Jacobi-–Anger expansion to Fourier transform the hopping amplitues:
\begin{equation}
\label{JacobiAnger}
e^{ie\bs{R}_{ij}\bs{A}(t)} = \sum_m e^{i(\frac{\pi}{2}-\theta_{ij})m} \mathcal{J}_m(\alpha_{ij}) e^{im\omega t} 
\end{equation}
where $\mathcal{J}_m(x)$ is the \textit{m}th Bessel function \cite{Kitamura2017}, modulating the weight of the \textit{m}th Fourier mode of the hopping amplitude. Let $\hat{T}_0$ be the hopping part of the hamiltonian, so that $\hat{H}_0 = \hat{T}_0 + \text{U}\hat{D}$. When an electric field is applied, the hopping amplitudes become time dependent so that $\hat{H}(t) = \hat{T}(t) +  \text{U}\hat{D}$. Using \ref{JacobiAnger} we can write $\hat{T}(t) = \sum_m \hat{T}_m e^{im \omega t}$ where $\hat{T}_m$ is the sum of all the \textit{m}th Fourier mode of the hopping terms. Additionaly we can further decompose the hopping operator into:

\begin{equation}
\hat{T}(t) = \sum_m (\hat{T}_{-1,m}+\hat{T}_{0,m}+\hat{T}_{1,m})e^{im\omega t}
\end{equation}

Where $\hat{T}_{dm}(t)$ changes the doublon number by $d$, for example, if $\hat{P}_d$ is the projection operator into the subspace with doublon number $d$, then $\hat{T}_{dm}(t) = \sum_i \hat{P}_{i+d}\hat{T}_{m}(t)\hat{P}_i$.

In order to derive the form of the effective Hamiltonian let us introduce a time dependent unitary transformation $\hat{U}(t) = e^{-i\hat{S}(t)}$. The transformed Hamiltonian is:

\begin{equation}
\hat{H}'(t) = e^{i\hat{S}(t)} \hat{H}(t) e^{-i\hat{S}(t)} - e^{i\hat{S}(t)} id_t e^{-i\hat{S}(t)}
\label{Htransformed}
\end{equation} 

We perform the unitary transformation perturbatively in the hopping operator, we can formally write $\hat{T}(t) = \eta \hat{T}(t)$, where $\eta$ will play the role of a bookkeeping parameter in the perturbative expansion. We expand $\hat{S}(t) = \sum_\nu \eta^\nu \hat{S}^{(\nu)}(t)$ and $\hat{H}'(t) = \sum_\nu \eta^\nu \hat{H}'^{(\nu)}(t)$. In order for the new Hamiltonian to be periodic we impose the unitary transformation to have the periodicity of the Hamiltonian, so that we can expand $\hat{S}^{(\nu)}(t) = \sum_m e^{im\omega t}\hat{S}^{(\nu)}_m$. Additionally we can impose the transformed Hamiltonian to be block diagonal in the doublon number $d$. With these conditions the unitary transformation can be uniquely determined if we impose that $\hat{S}(t)$ does not contain block-diagonal terms, i.e. we can write:
\begin{equation}
\hat{S}^{(\nu)}(t) = \sum_{d \neq 0} \sum_m \eta^\nu \hat{S}^{(\nu)}_{d,m} e^{im\omega t}
\end{equation}
where $\hat{S}^{(\nu)}_{d,m}$ changes the double occupancy number by $d$. From here the procedure is simple but lenghty, we expand \ref{Htransformed} in power series of $\eta$ and determine $\hat{S}^{(\nu)}(t)$ iteratively in $\nu$ so that $\hat{H}'^{(\nu)}(t)$ is diagonal in the doublon number. Up to second order we obtain:
\begin{align}
&\hat{H}'(t) \approx \text{U}\hat{D} - \sum_m \hat{T}_{0,m}(t)e^{im\omega t} + \n \\
&+ \frac{1}{2}\sum_{mn} \left( \frac{\left[\hat{T}_{1n}, \hat{T}_{-1(m-n)} \right]}{\text{U}+n\omega} - \frac{\left[\hat{T}_{-1n}, \hat{T}_{1(m-n)} \right]}{\text{U}-n\omega} \right) e^{im\omega t}
\end{align}

Let $\hat{P}_0$ be the projection operator to the $d=0$ subspace. Then notice that $\hat{P}_0\hat{H}'^{1}(t)\hat{P}_0 = 0$, we define $\hat{H}_{\text{eff}}$ to be the time average of $\hat{P}_0\hat{H}'^{2}(t)\hat{P}_0$. Expressed in terms of creation and anihilation operators we obtain:

\begin{align}
\hat{H}_{\text{eff}} &= - \sum_{i,j, \sigma, \sigma'} \left\{ \hat{c}_{i \sigma}^\dagger \hat{c}_{j \sigma} \hat{c}_{j \sigma'}^\dagger \hat{c}_{i \sigma'} \right. \n \\
&t_{ij}^{\sigma} t_{ji}^{\sigma'} \sum_{n} \frac{\mathcal{J}_{n}^2(\alpha_{ij})}{\text{U}+n\omega} \left. \right\} \label{GeneralHeff}
\end{align}

Where $t_{ij}^{\sigma}$ are the corresponding amplitudes for the unperturbed Hamiltonian, i.e. $t_{ij}^{\sigma} = t_1$ for $i,j$ being NN and $t_{ij}^{\sigma} = t_2 + i\Delta\nu_{ij}\sigma^z_{\sigma, \sigma}$ for $i,j$ being NNN. We can obtain the spin Hamiltonian by using the relations:
\begin{align}
\hat{c}_{i \sigma}^\dagger \hat{c}_{i \sigma'} &= \delta_{\sigma \sigma'} \frac{1}{2} (n_{i \uparrow} + n_{i \downarrow}) + \boldsymbol{S}_i\boldsymbol{\sigma}_{\sigma', \sigma} \label{SpinOperatorInv1}\\ 
\hat{c}_{i \sigma} \hat{c}_{i \sigma'}^\dagger &= \delta_{\sigma \sigma'} \frac{1}{2} (2 - n_{i \uparrow} - n_{i \downarrow}) - \boldsymbol{S}_i\boldsymbol{\sigma}_{\sigma, \sigma'} \label{SpinOperatorInv2}
\end{align}
and $n_{i \uparrow} + n_{i \downarrow} = 1$. 
and summing over the spin states. The resulting effective spin Hamiltonian is:

\begin{align}
\label{MKMHeff}
&\hat{H}_{\text{eff}} = \sum_{\langle i,j \rangle} J_{1,ij}\bs{S}_i\bs{S}_j +\n \\
&+ \sum_{\langle \langle i,j \rangle \rangle} \left\{ J_{2,ij}\bs{S}_i\bs{S}_j + \bs{D}_{2,ij} \bs{S}_i \times \bs{S}_j + \bs{S}_i \bs{\Gamma}_{ij} \bs{S}_j \right\}
\end{align}

Where:

\begin{align*}
J_{1,ij} &= 2t_1^2\frac{\mathcal{J}_{n}^2(\alpha_{ij})}{\text{U}+n\omega} \\
J_{2,ij} &= 2t_2^2\frac{\mathcal{J}_{n}^2(\alpha_{ij})}{\text{U}+n\omega} \\
\bs{D}_{2,ij} &= - 4\nu_{ij} t_2 \Delta \hat{e}_z \frac{\mathcal{J}_{n}^2(\alpha_{ij})}{\text{U}+n\omega} \\
\bs{\Gamma}_{2,ij} &= 2\Delta^2 \text{diag}(-1,-1,1) \frac{\mathcal{J}_{n}^2(\alpha_{ij})}{\text{U}+n\omega}
\end{align*}

Now, a Hamiltonian with the form $\hat{H} = \sum_{\langle i,j \rangle} J_1 \bs{S}_i\bs{S}_j + \sum_{\langle \langle i,j \rangle \rangle} J_2\bs{S}_i\bs{S}_j$ is known as the $J_1$-$J_2$ Heisenberg model and in a 2D honeycomb lattice it exhibits N\'eel order for $J_2 < J_1 / 6$ and for $J_2 > J_1 / 6$ spin density waves (SDW) appear \cite{Mulder2010}. In the presence of DMI alone there will always be SDW in the plane perpendicular to $\bs{D}$ \cite{Uchida2006}. In Hamiltonian \ref{MKMHeff} we expect SDW to appear in the ground state and the SDW wavevector will be determined by a function of the parameters of this model. In the next section we will do a numerical of study on how modifying the ratio between NN and NNN spin interaction factors can change the spin state of the system.

\begin{subsection}{Disorder}
It is possible to introduce the effect of disorder in the model \ref{MKMH} by adding random uncorrelated on-site energies:

\begin{align}
\label{DisorderedHubbardModel}
\hat{H}_0 &= - \sum_{\langle i,j \rangle, \sigma} t_1\hat{c}_{i \sigma}^\dagger \hat{c}_{j \sigma} -\sum_{\langle \langle i,j \rangle \rangle, \sigma}(t_2 - i\Delta\nu_{ij}\sigma^z_{\sigma, \sigma})\hat{c}_{i \sigma}^\dagger \hat{c}_{j \sigma} \n \\
	& + \sum_{i \sigma} \epsilon_i \hat{c}_{i \sigma}^\dagger \hat{c}_{i \sigma} +
	\text{U}\hat{D}
\end{align}

Where $\epsilon_i$ are uncorrelated random variables $\epsilon_i \in [-W,W]$. At half filling we can derive an effective Hamiltonian using the same procedure as before. The second order virtual hopping $i\sigma \rightarrow j\sigma \rightarrow i\sigma$ will give rise to an exchange spin interaction. In this case, however, the intermediate energy is $\text{U} + (\epsilon_j - \epsilon_i)$. Therefore the spin Hamiltonian will be:

\begin{align}
&\hat{H}_{\text{eff}}^{\text{Dis}} = \sum_{\langle i,j \rangle} J_{1,ij}\bs{S}_i\bs{S}_j +\n \\
&+ \sum_{\langle \langle i,j \rangle \rangle} \left\{ J_{2,ij}\bs{S}_i\bs{S}_j + \bs{D}_{2,ij} \bs{S}_i \times \bs{S}_j + \bs{S}_i \bs{\Gamma}_{ij} \bs{S}_j \right\}
\end{align}

With:

\begin{align*}
J_{1,ij} &= \frac{2t_1^2\text{U}}{\text{U}^2-(\epsilon_j-\epsilon_i)^2} \\
J_{2,ij} &= \frac{2t_2^2\text{U}}{\text{U}^2-(\epsilon_j-\epsilon_i)^2} \\
\bs{D}_{2,ij} &= -\frac{4\nu_{ij} t_2 \Delta \text{U}}{\text{U}^2-(\epsilon_j-\epsilon_i)^2}\hat{e}_z \\
\bs{\Gamma}_{2,ij} &= \frac{2\Delta^2\text{U}}{\text{U}^2-(\epsilon_j-\epsilon_i)^2}\text{diag}(-1,-1,1)
\end{align*}

Where in the second step we added the contributions of $\langle i,j \rangle$ and $\langle j,i \rangle$ and where $J_{ij} = \frac{2t^2\text{U}}{\text{U}^2-(\epsilon_j-\epsilon_i)^2}$. This model is relevant for studying many-body localization phenomena \cite{Protopopov}.
\end{subsection}

\end{section}

\section*{Acknowledgments}

%\begin{thebibliography}
\bibliography{paper}
%\end{thebibliography}

\end{document}


