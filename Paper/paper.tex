% !TEX spellcheck = en-US
\documentclass[aps,prl,twocolumn,amsmath,amssymb,superscriptaddress,nobibnotes]{revtex4}%
\usepackage{graphicx}
\usepackage{dcolumn}
\usepackage{bm}
\usepackage{color}
\usepackage{amsmath}
\usepackage{epstopdf}
%\usepackage{caption}
%\usepackage{subcaption}
\usepackage{fancybox}
\usepackage{amsfonts}
\usepackage{amssymb}%
\usepackage[toc,page]{appendix}
%\usepackage[backend=bibtex,natbib=true,sorting=none]{biblatex}
\setcounter{MaxMatrixCols}{30}


\renewcommand{\cite}[1]{{[}\onlinecite{#1}{]}}

\newcommand{\s}{\sum\limits}
\newcommand{\p}{\prod\limits}
\newcommand{\pa}{\partial}
\newcommand{\il}{\int\limits}
\newcommand{\be}{\begin{equation}}
\newcommand{\e}{\end{equation}}
\newcommand{\beml}{\begin{subequations}}
\newcommand{\eml}{\end{subequations}}
\newcommand{\beq}{\begin{eqnarray}}
\newcommand{\eq}{\end{eqnarray}}
\newcommand{\ba}{\begin{array}}
\newcommand{\ea}{\end{array}}
\newcommand{\bpm}{\begin{pmatrix}}
\newcommand{\epm}{\end{pmatrix}}
\newcommand{\bc}{\begin{cases}}
\newcommand{\ec}{\end{cases}}
\newcommand{\lt}{\left}
\newcommand{\rt}{\right}
\newcommand{\n}{\nonumber}
\newcommand{\la}{\langle}
\newcommand{\ra}{\rangle}
\newcommand{\ep}{\varepsilon}
\newcommand{\dd}{\displaystyle}
\newcommand{\bs}{\boldsymbol}
\newcommand{\h}{^\dagger}
\newcommand{\ph}{^{\phantom{\dagger}}}
\newcommand{\one}{\openone}
\newcommand{\ut}{\underaccent{\tilde}}
\newcommand{\ul}{\underaccent{\bar}}
\newcommand{\up}{\uparrow}
\newcommand{\dn}{\downarrow}
\DeclareMathOperator{\var}{var}
\DeclareMathOperator{\tr}{Tr}
\DeclareMathOperator{\diag}{diag}
\DeclareMathOperator{\im}{Im}
\DeclareMathOperator{\re}{Re}
\DeclareMathOperator{\ctg}{cotan}
\DeclareMathOperator{\arcsh}{arcsinh}
\DeclareMathOperator{\csch}{csch}
\DeclareMathOperator{\rot}{rot}

\newcommand{\AQ}[1]{\textbf{\textcolor{red}{{#1}}}}

\begin{document}
\title{Ultrafast manipulation of Heisenberg exchange and Dzyaloshinskii–Moriya interactions in antiferromagnetic insulators}


\begin{abstract}
This is the abstract.
\end{abstract}

\date{\today}
\maketitle

\begin{section}{Model and calculations}

We will start we a general Hamiltonian:

\begin{equation}
\hat{H} = -\sum_{ij\sigma \sigma'} t_{ij}^{\sigma \sigma'} \hat{c}_{i \sigma}^\dagger \hat{c}_{j \sigma'} + \text{U}\hat{D}
\end{equation}

Where $\hat{c}_{i \sigma}^\dagger$ ($ \hat{c}_{i \sigma}$) creates (anihilates) an electron at site $i$ in spin state $\sigma$, $t_{ij}^{\sigma \sigma'}$ are the hopping amplitudes originating from kinetik terms or spin orbit interaction. $\hat{D} = \sum_{i=1}^M \hat{n}_{i\uparrow}\hat{n}_{i\downarrow}$ is the doublon number operator and $\text{U}$ is the on-site interaction. We will assume a half filling system in which the strength of the on-site interaction $\text{U}$ is much larger than the hopping amplitudes, therefore the zero double occupancies subspace $d=0$ can be taken as the low energy subspace in which the effective Hamiltonian will act.

The hopping term $-\hat{T} = -\sum_{ij\sigma \sigma'} t_{ij}^{\sigma \sigma'} \hat{c}_{i \sigma}^\dagger \hat{c}_{j \sigma'}$ can describe various models. We will be interested in:

\begin{itemize}
	\item The Hubbard model, in which the hopping does not change the spin, $t_{ij}^{\sigma \sigma'} = \delta_{\sigma \sigma'}t$ for $i,j$ NN and $t_{ij}^{\sigma \sigma'} = 0$ otherwise.
	\item The so-called Kane-Mele-Hubbard model, which includes a spin-dependent term araising from spin-obit interactions.
	\begin{equation}
t_{ij}^{\sigma \sigma'} = \begin{cases}
	\delta_{\sigma \sigma'}t_1 & \text{for } i, j \text{ nearest neighbors} \\
	-\Delta \nu_{ij} \hat{e}_z \cdot \bs{\sigma}_{\sigma, \sigma'} & \text{for } i, j \text{ next nearest neighbors} \\
	0 & \text{ otherwise} \n
\end{cases} \quad
\end{equation}
\end{itemize}

In the prescence of an laser perturbation we can use the Peirls substitution to introduce the effect of the field trough the hopping amplitudes. We can write the electric field as $\bs{E}(t) = \frac{1}{2}(\vec{E}e^{-i\omega t}+\vec{E}^*e^{i\omega t})$, $\vec{E} = E_0\hat{e}$ and $\hat{e} = \frac{1}{\sqrt{1+\lambda_{POL}^2}}(\hat{e}_x+i\lambda_{POL}\hat{e}_y)$ is the polarization vector and $\lambda_{POL} = 0, \pm 1$ for plane polarized, right handed and left handed circular polarized field respectively. According to the Peierls rule the hopping amplitudes gain a phase:
\begin{equation}
t_{ij}^{\sigma \sigma'}(t) = t_{ij}^{\sigma \sigma'} e^{ie\bs{R}_{ij}\bs{A}(t)}
\end{equation}
where $\bs{R}_{ij} = \bs{R}_i-\bs{R}_j$, $\bs{R}_i$ is the position of site $i$ and $\bs{A}$ is the vector potential $\bs{A}(t) = \frac{1}{2}(\vec{A}e^{-i\omega t} + \vec{A}^* e^{i\omega t})$, with $\vec{A} = \frac{iE_0}{\omega}\hat{e}$.

Let us define:

\begin{equation}
\label{Def_alpha}
e\bs{R}_{ij}\vec{A} = \alpha_{ij} e^{i \theta_{ij}}
\end{equation}

With $\alpha_{ij} = \pm|e\bs{R}_{ij}\vec{A}|$ in such a way that:

\begin{align}
\alpha_{ij} &= -\alpha_{ji} \label{alphaSym} \\
\theta_{ij} &= \theta_{ji} \label{thetaSym}
\end{align}

and $\theta_{ij} \in \left[0,\pi\right)$. Then we can apply the Jacobi-–Anger expansion:

\begin{align}
t_{ij}^{\sigma \sigma'}(t) &= t_{ij}^{\sigma \sigma'}\sum_m e^{i(\frac{\pi}{2}-\theta_{ij})m} \mathcal{J}_m(\alpha_{ij}) e^{im\omega t} = \n \\
&= \sum t_{ij,m}^{\sigma \sigma'} e^{im\omega t}
\end{align}

Where we defined $t_{ij,m}^{\sigma \sigma'} = t_{ij}^{\sigma \sigma'} e^{i(\frac{\pi}{2}-\theta_{ij})m} \mathcal{J}_m(\alpha_{ij})$, which is the \textit{m}th Fourier mode of the hopping term and $\mathcal{J}_m(x)$ is the \textit{m}th Bessel function \cite{Kitamura2017}. In order to derive the form of the effective Hamiltonian let us introduce a time dependent unitary transformation $\hat{U}(t) = e^{-i\hat{S}(t)}$. The transformed Hamiltonian is:

\begin{equation}
\hat{H}'(t) = e^{i\hat{S}(t)} \hat{H}(t) e^{-i\hat{S}(t)} - e^{i\hat{S}(t)} id_t e^{-i\hat{S}(t)}
\end{equation} 

We perform the unitary transformation perturbatively in the hopping operator, we can formally write $\hat{T}(t) = \eta \hat{T}(t)$, where $\eta$ will play the role of a bookkeeping parameter in the perturbative expansion. We expand $\hat{S}(t) = \sum_\nu \eta^\nu \hat{S}^{(\nu)}(t)$ and $\hat{H}'(t) = \sum_\nu \eta^\nu \hat{H}'^{(\nu)}(t)$. In order for the new Hamiltonian to be periodic we impose the unitary transformation to have the periodicity of the Hamiltonian, so that we can expand $\hat{S}^{(\nu)}(t) = \sum_m e^{im\omega t}\hat{S}^{(\nu)}_m$. Additionally we can impose the transformed Hamiltonian to be block diagonal in the doublon number $d$. With these conditions the unitary transformation can be uniquely determined if we impose that $\hat{S}(t)$ does not contain block-diagonal terms, i.e. we can write:
\begin{equation}
\hat{S}^{(\nu)}(t) = \sum_{d \neq 0} \sum_m \eta^\nu \hat{S}^{(\nu)}_{d,m} e^{im\omega t}
\end{equation}
where $\hat{S}^{(\nu)}_{d,m}$ changes the double occupancy number by $d$.

Let $\hat{P}_0$ be the projection operator to the $d=0$ subspace. Then we obtain $\hat{P}_0\hat{H}'^{1}(t)\hat{P}_0 = 0$, we define $\hat{H}_{\text{eff}}$ to be the time average of $\hat{P}_0\hat{H}'^{2}(t)\hat{P}_0$. We obtain:

\begin{align}
\hat{H}_{\text{eff}} &= - \sum_{i,j, \sigma_1, \sigma_2, \sigma_3, \sigma_4} \left\{ \hat{c}_{i \sigma_1}^\dagger \hat{c}_{j \sigma_2} \hat{c}_{j \sigma_3}^\dagger \hat{c}_{i \sigma_4} \right. \n \\
&t_{ij}^{\sigma_1 \sigma_2} t_{ji}^{\sigma_3 \sigma_4} \sum_{n} \frac{\mathcal{J}_{n}^2(\alpha_{ij})}{\text{U}+n\omega} \left. \right\} \label{GeneralHeff}
\end{align}

We can obtain the spin Hamiltonian by using the relations:
\begin{align}
\hat{c}_{i \sigma}^\dagger \hat{c}_{i \sigma'} &= \delta_{\sigma \sigma'} \frac{1}{2} (n_{i \uparrow} + n_{i \downarrow}) + \boldsymbol{S}_i\boldsymbol{\sigma}_{\sigma', \sigma} \label{SpinOperatorInv1}\\ 
\hat{c}_{i \sigma} \hat{c}_{i \sigma'}^\dagger &= \delta_{\sigma \sigma'} \frac{1}{2} (2 - n_{i \uparrow} - n_{i \downarrow}) - \boldsymbol{S}_i\boldsymbol{\sigma}_{\sigma, \sigma'} \label{SpinOperatorInv2}
\end{align}
and $n_{i \uparrow} + n_{i \downarrow} = 1$.

\begin{subsection}{Kane--Mele--Hubbard model}

The first model to describe topological insulators was introduced by Kane and Mele \cite{Kane2005} to describe quantum spin Hall effect in graphene. In a honeycomb lattice time reversal symmetry and inversion symmetry allow only next-nearest neighbor spin orbit coupling, which is known as intrinsic spin orbit coupling. In these circumstances the system can be modeled by the Kane-Mele-Hubbard model, which has hopping amplitudes:

\begin{equation}
t_{ij}^{\sigma \sigma'} = \begin{cases}
	\delta_{\sigma \sigma'}t_1 & \text{for } i, j \text{ nearest neighbors} \\
	-\Delta \nu_{ij} \hat{e}_z \cdot \bs{\sigma}_{\sigma, \sigma'} & \text{for } i, j \text{ next nearest neighbors} \\
	0 & \text{ otherwise} \n
\end{cases} \quad
\end{equation}

Where $\Delta$ is the intrinsic spin orbit coupling constant. $\nu_{ij}=\pm 1$ depending on whether the electron traversing from $i$ to $j$ makes a right ($+1$) or a left turn ($-1$). Then, introducing the spin operators and the corresponding hopping amplitudes in (\ref{GeneralHeff}) we obtain the effective spin Hamiltonian of this model:

\begin{equation}
\label{KMHeff}
\hat{H}_{\text{KMH}}^{\text{eff}} = \sum_{\langle i,j \rangle} J_{ij} \bs{S}_i \bs{S}_j + \sum_{\langle \langle i,j \rangle \rangle} \bs{S}_i \bs{\Gamma}_{ij} \bs{S}_j 
\end{equation}

Where the exchange interaction couples NN spins and an anisotropic exchange interaction couples NNN spins:

\begin{align*}
J_{ij} &= 2t_1^2\frac{\mathcal{J}_{n}^2(\alpha_{ij})}{\text{U}+n\omega} \\
\bs{\Gamma}_{ij} &= 2\Delta^2 \text{diag}(-1,-1,1)\frac{\mathcal{J}_{n}^2(\alpha_{ij})}{\text{U}+n\omega}
\end{align*}
\end{subsection}

\begin{subsection}{Modified Kane--Mele--Hubbard model}

As we saw in the previous section the effective Hamiltonian \ref{KMHeff} does not contain NNN DMI terms. The reason for this is because DMI originates from the virtual hoppings $i\rightarrow j\rightarrow i$ combining SOI hopping and kinetik hopping, since in the original Kane-Mele model we don't find NNN kinetik hopping, the DMI term vanishes. Next we will study the same Hamiltonian adding a finite NNN hopping term $t_2$, which can be understood as a second order NN hopping process (the direct NNN hopping integral would usually be much smaller). Thus we start with:

\begin{align}
\label{MKMH}
\hat{H} &= - t_1\sum_{\langle i,j \rangle, \sigma} \hat{c}_{i \sigma}^\dagger \hat{c}_{j \sigma} + \n \\
	&+\sum_{\langle \langle i,j \rangle \rangle, \sigma}(t_2 + i\Delta\nu_{ij}\sigma^z_{\sigma, \sigma})\hat{c}_{i \sigma}^\dagger \hat{c}_{j \sigma} + 
	\text{U}\hat{D}
\end{align}

Proceeding in the same way we find the effective spin Hamiltonian:

\begin{align}
\label{MKMHeff}
&\hat{H}_{\text{eff}}(t) = \sum_{\langle i,j \rangle} J_{1,ij}\bs{S}_i\bs{S}_j +\n \\
&+ \sum_{\langle \langle i,j \rangle \rangle} \left\{ J_{2,ij}\bs{S}_i\bs{S}_j + \bs{D}_{2,ij} \bs{S}_i \times \bs{S}_j + \bs{S}_i \bs{\Gamma}_{ij} \bs{S}_j \right\}
\end{align}

Where:

\begin{align*}
J_{1,ij} &= 2t_1^2\frac{\mathcal{J}_{n}^2(\alpha_{ij})}{\text{U}+n\omega} \\
J_{2,ij} &= 2t_2^2\frac{\mathcal{J}_{n}^2(\alpha_{ij})}{\text{U}+n\omega} \\
\bs{D}_{2,ij} &= - 4\nu_{ij} t_2 \Delta \hat{e}_z \frac{\mathcal{J}_{n}^2(\alpha_{ij})}{\text{U}+n\omega} \\
\bs{\Gamma}_{2,ij} &= 2\Delta^2 \text{diag}(-1,-1,1) \frac{\mathcal{J}_{n}^2(\alpha_{ij})}{\text{U}+n\omega}
\end{align*}

Now, a Hamiltonian with the form $\hat{H} = \sum_{\langle i,j \rangle} J_1 \bs{S}_i\bs{S}_j + \sum_{\langle \langle i,j \rangle \rangle} J_2\bs{S}_i\bs{S}_j$ is known as the $J_1$-$J_2$ Heisenberg model and in a 2D honeycomb lattice it exhibits N\'eel order for $J_2 < J_1 / 6$ and for $J_2 > J_1 / 6$ spin density waves (SDW) appear \cite{Mulder2010}. In the presence of DMI alone there will always be SDW in the plane perpendicular to $\bs{D}$ \cite{Uchida2006}. In Hamiltonian \ref{MKMHeff} we expect SDW to appear in the ground state and the SDW wavevector will be determined by a function of the parameters of this model. In the next section we will do a numerical of study on how modifying the ratio between NN and NNN spin interaction factors can change the spin state of the system.

\end{subsection}

\end{section}

\section*{Acknowledgments}

%\begin{thebibliography}
\bibliography{paper}
%\end{thebibliography}

\end{document}


